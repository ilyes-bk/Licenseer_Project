\section{Conclusions}
\label{Section:Conclusions}

\textcolor{black}{This paper presents LARK (License Analysis with RAG and Knowledge graphs), a novel framework that addresses critical challenges in software license compatibility detection and regulatory compliance. Modern software development increasingly relies on complex ecosystems of open-source components, where 72.91\% of projects encounter license incompatibilities that threaten project viability and regulatory compliance. Our methodology integrates Knowledge Graphs, Large Language Models, and Retrieval-Augmented Generation to provide comprehensive license compatibility analysis with explainable reasoning.

\textbf{Technical Contributions:} Our framework makes several key technical contributions: (1) \textit{Integrated Knowledge Representation} through Neo4j-based knowledge graphs that model 750+ licenses, 20,000+ dependencies, and their compatibility relationships enabling transitive analysis across complex dependency chains; (2) \textit{LLM-Powered Custom License Integration} using GPT-4 with few-shot learning to automatically parse and integrate novel license texts without requiring model retraining; (3) \textit{RAG-Enhanced Explainability} that provides detailed, citation-backed explanations with 92.8\% retrieval precision from 25,000+ indexed segments across comprehensive legal literature including authoritative books, IEEE/ACM articles, and regulatory guidelines, providing 3.2x more citations than existing approaches with exact page numbers and section references; (4) \textit{Dual-Constraint Hallucination Mitigation} through knowledge graph constraints that prevent LLM decision-making and RAG grounding that ensures all explanations reference authoritative legal sources, achieving 2.1\% hallucination rate compared to 18.7\% for unconstrained LLM responses; (5) \textit{Dynamic Update Capabilities} supporting rapid adaptation to evolving legal interpretations through graph-based knowledge representation.

\textbf{Experimental Validation:} Our comprehensive evaluation on 4,000 OSS projects demonstrates significant improvements over existing approaches. LARK achieves (1) 98.1\% license detection accuracy compared to 93.2\% for LiDetector; (2) 96.2\% conflict detection F1 score versus 88.7\% for baselines; (3) 4.8/5 explainability score compared to 3.2/5 for existing tools; (4) 94\% accuracy in processing custom and proprietary licenses where traditional tools provide 23\% capability; (5) 2.1\% hallucination rate compared to 18.7\% for unconstrained LLM responses, with 97.8\% verifiable citations; (6) 24-hour update cycles versus 168 hours for ML-based approaches; (7) 0.32GB memory usage representing 71\% reduction compared to existing systems.

\textbf{Practical Impact:} The framework addresses critical gaps in existing approaches: limited explainability in current tools, static knowledge representation requiring manual updates, narrow compatibility analysis focus, and poor custom license handling. LARK provides citation-backed explanations essential for regulatory compliance, supports automatic integration of custom licenses, and enables real-time compatibility monitoring through CI/CD integration.

\textbf{Future Work:} We plan to extend coverage to emerging license formats, integrate additional package managers, develop regulatory framework-specific policies (GDPR, CCPA), and create industry-specific knowledge graphs for domain-specific compliance requirements. Further research directions include fine-tuning specialized legal LLMs, developing temporal license evolution models, and exploring multi-modal analysis combining license text with code patterns.

We believe that our results contribute to a significant advancement in automated license compatibility detection, providing developers and organizations with accurate, explainable, and scalable solutions for legal risk mitigation and regulatory compliance in modern software engineering.

%\noindent \textbf{Acknowledgments.} We would like to thank the reviewers at IST for their detailed and invaluable feedback.

\noindent \textbf{Data Availability Statement.} The data is publicly available at \cite{ReplicationPackage}.

\noindent{\textbf{Declaration of generative AI and AI-assisted technologies in the writing process.}}



\section{Acknowledgements}
\label{sec:Acknowledgements}
We would like to thank our industrial partner Vermeg-Tunisia for providing the resources and support necessary to conduct this research. Their assistance in 
coordinating the data acquisition and labeling process was invaluable.
This study has been funded by the Tunisian Young Researchers’ Encouragement Program (Ed. 2022) (22PEJC-D3P2).\\

\section{Conflict of interest}
All authors in this paper declared that there is no conflicts of interest to this work.

