
%\documentclass[5p,times]{elsarticle}
\documentclass[review]{elsarticle}


\AtBeginDocument{\renewcommand*{\thesubfigure}{\alphalph{\value{subfigure}}}}
\renewcommand{\footnotesize}{\scriptsize}
\usepackage{textcomp}
\usepackage{pgfplots, pgfplotstable}
\usepackage{pgf-pie}
\usepackage{graphicx}
\usepackage{bbding}
\usepackage{xcolor}
\usepackage{enumitem}
\usepackage{hyperref}
\usepackage{cleveref}
\usepackage{lmodern,textcomp}
%\usepackage{subfig}
%\usepackage{caption}
%\usepackage{subcaption}
%\usepackage{subfigure}
\usepackage{longtable}
%\usepackage{subfigure}
\usepackage{subcaption}
\usepackage{alphalph}
\usepackage[many]{tcolorbox}
\usepackage{xcolor}
\biboptions{sort&compress}
%\usepackage[table]{xcolor}
\usepackage{array}
\usepackage{colortbl}
\newtheorem{hypothesis}{Hypothesis}
\newtheorem{nullhypothesis}{Null Hypothesis}
\newcolumntype{L}{>{\arraybackslash}m{16cm}}
\usepackage[flushleft]{threeparttable}
%\newcolumntype{L}[1]{>{\raggedright\let\newline\\arraybackslash\hspace{0pt}}m{#1}}
\newcolumntype{C}[1]{>{\centering\let\newline\\arraybackslash\hspace{0pt}}m{#1}}
\newcolumntype{R}[1]{>{\raggedleft\let\newline\\arraybackslash\hspace{0pt}}m{#1}}
\def\BibTeX{{\rm B\kern-.05em{\sc i\kern-.025em b}\kern-.08em
    T\kern-.1667em\lower.7ex\hbox{E}\kern-.125emX}}
\usepackage{csquotes}

\usepackage{tikz}
 %% footnote fonts
\usetikzlibrary{fit} %% Used for putting dotted box in image
%\usepackage[margin=2cm]{geometry}
%\usepackage{beamerposter}
\usetikzlibrary{positioning}
\usetikzlibrary{arrows}
\usetikzlibrary{shapes.multipart}

\usepackage[hidelinks,bookmarks=false]{hyperref}
%\usepackage[numbered]{bookmark}
\usepackage{soul}
\usepackage{booktabs}
\usepackage{multirow}
\usepackage{float}
\usepackage{url}
%\usepackage[small,it]{caption}
%\usepackage{tcolorbox}
%\usepackage{cite}
\usepackage{amsmath,amssymb,amsfonts}
\usepackage{algorithmic}
%\usepackage{graphicx}
\usepackage{textcomp}
%\usepackage{xcolor}
\usepackage{url}
%\usepackage{tabularx} 
%\usepackage{lineno,hyperref}
\usepackage{longtable}
%\modulolinenumbers[5]
\journal{Journal of \LaTeX\ Templates}
 
%\usepackage{tabularx}
\usepackage{pgfplots}
\pgfplotsset{width=7cm,compat=1.8,tick label style={font=\small}}


\usepackage{xspace}
\newcommand{\ie}{\textit{i.e., \xspace}}
\newcommand{\eg}{\textit{e.g., \xspace}}
\newcommand{\etal}{\textit{et al. \xspace}}
\newcommand{\bluetext}[1]{\textcolor{blue}{#1}}
%\usepackage{graphicx}
\usepackage{adjustbox}
\usepackage{float}
\usepackage{rotating}
\usepackage{tablefootnote}
\usepackage{nth}
\DeclareCaptionType{TextBox}
%\newcolumntype{L}{>{\centering\arraybackslash}m{3cm}}
\newcommand{\taxonomy}{25\xspace}
\usepackage[utf8]{inputenc}
\usepackage{dirtytalk}
\usepackage{tcolorbox}

\usepackage[british]{babel}
\usepackage{enumitem}

\newlist{SubItemList}{itemize}{1}
\setlist[SubItemList]{label={$-$}}

\let\OldItem\item
\newcommand{\SubItemStart}[1]{%
    \let\item\SubItemEnd
    \begin{SubItemList}[resume]%
        \OldItem #1%
}
\newcommand{\SubItemMiddle}[1]{%
    \OldItem #1%
}
\newcommand{\SubItemEnd}[1]{%
    \end{SubItemList}%
    \let\item\OldItem
    \item #1%
}
\newcommand*{\SubItem}[1]{%
    \let\SubItem\SubItemMiddle%
    \SubItemStart{#1}%
} 

%\usepackage{fontawesome}


\bibliographystyle{elsarticle-num}

  \pgfplotstableread[row sep=\\,col sep=&]{
        interval & carT & sd\\
      %  A & 2.3  \\
      %  B & 18.3  \\
      %  C & 41.9 \\
      %  D & 6.3  \\
      %  E & 20.6 \\
      %  F & 10.6 \\
     }\mydata

\newtcolorbox{boxK}{
    sharpish corners, % better drop shadow
    boxrule = 0pt,
    toprule = 4.5pt, % top rule weight
    enhanced,
    fuzzy shadow = {0pt}{-2pt}{-0.5pt}{0.5pt}{black!35} % {xshift}{yshift}{offset}{step}{options} 
}

%%%%%%%%%%%%%%%%%%%%%%%%%%%%%%%%%%%%%
%%%%%%%%%%_START: Flowchart Setup Code__%%%%%%%%%%
% source: https://www.sharelatex.com/project/52205bbce77a8bec1415bf38
%\usepackage{tikz}
\usetikzlibrary{shapes.geometric, arrows} 

\tikzstyle{startstop} = [rectangle, rounded corners, minimum width=3cm, minimum height=1cm,text centered, draw=black, fill=black!10]

\tikzstyle{io} = [trapezium, trapezium left angle=70, trapezium right angle=110, minimum width=3cm, minimum height=1cm, text centered,text width=2cm, draw=black]

\tikzstyle{process} = [rectangle, minimum width=3cm, minimum height=1cm, text centered, text width=3cm, draw=black]

\tikzstyle{decision} = [diamond, minimum width=3cm, minimum height=1cm, text centered,text width=2cm, draw=black]

\tikzstyle{arrow} = [thick,->,>=stealth]

\tikzstyle{database} = [cylinder, shape border rotate=90, draw=black,minimum height=2cm,minimum width=3cm, text centered,text width=0.6cm]
%%%%%%%%%%_END: Flowchart Setup Code__%%%%%%%%%%

\newcommand{\anthony}[1]{\textcolor{blue}{{\it [Anthony says: #1]}}}
\newcommand{\ali}[1]{\textcolor{red}{{\it [Ali: #1]}}}
\newcommand{\eman}[1]{\textcolor{violet}{{\it [Eman says: #1]}}}
\newcommand{\mohamed}[1]{\textcolor{purple}{{\it [Mohamed says: #1]}}}
\newcommand{\steven}[1]{\textcolor{orange}{{\it [Steven says: #1]}}}
\newcommand{\christian}[1]{\textcolor{green}{{\it [Christian says: #1]}}}

\newcommand{\RQA}{RQ$_1$: How have refactoring discussions on Stack Overflow grown over the years?}
\newcommand{\RQAA}{RQ$_{1.1}$: How have refactoring posts grown throughout the years?}
\newcommand{\RQAB}{RQ$_{1.2}$: What is the distribution of questions and answers among developers?}
\newcommand{\RQAC}{RQ$_{1.3}$: What are the tags that are associated with refactoring questions?}
\newcommand{\RQB}{RQ$_2$: What do developers discuss in refactoring based Stack Overflow posts?}
\newcommand{\RQBA}{RQ$_{2.1}$: What are the frequent terms utilized by developers in refactoring discussions?} 
%\newcommand{\RQBB}{RQ$_{2.2}$: To what extent do Self-Affirmed Refactoring in software artifacts match with the challenges faced by developers in Stack Overflow posts?}
\newcommand{\RQBB}{RQ$_{2.2}$: To what extent do traditional refactoring opportunities, known in existing literature, match with the challenges faced by developers in Stack Overflow posts?}
\newcommand{\RQBC}{RQ$_{2.3}$: What are the topics around software refactoring that are being asked by developers?} 
\newcommand{\RQC}{RQ$_3$: Which topics are the most popular and difficult among refactoring-related questions?} 


\begin{document}

%\newpage
%Hello! This paper has been accepted for publication at the Journal of Systems and Software. To cite the paper:

%AlOmar, Eman Abdullah, Mohamed Wiem Mkaouer, Ali Ouni. "Toward the Automatic Classification of Self-Affirmed Refactoring." Journal of Systems and Software (2020): to appear.

%Enjoy!

%\newpage


\begin{frontmatter}

\title{LARK: License Analysis with RAG and Knowledge graphs}


%% Group authors per affiliation:
\author[TBS]{Ilyes Ben Khalifa}
\ead{ilyes.benkhalifa@tbs.u-tunis.tn}


\author[LAR]{Montassar Ben Messaoud\corref{mycorrespondingauthor}}
\cortext[mycorrespondingauthor]{Corresponding author}
\ead{montassar.benmessaoud@tbs.u-tunis.tn}


\author[UoM]{Mohamed Wiem Mkaouer}
\ead{mmkaouer@umich.edu}




\address[LAR]{University of Tunis, LARODEC, Tunis Business School, El Mourouj, Tunisia}
\address[TBS]{University of Tunis, Tunis Business School, El Mourouj, Tunisia}
\address[UoM]{University of Michigan Flint, MI, USA}










\begin{abstract}


\noindent\textbf{Context:} Software systems increasingly rely on large ecosystems of libraries, frameworks, and platforms. While this interconnectedness accelerates development, it also gives rise to the persistent challenge of software license incompatibility. Rapid evolution of dependencies, heterogeneous execution environments, and the risk of deprecated or inconsistent APIs can lead to hidden costs, degraded reliability, and even security vulnerabilities. \noindent\textbf{Objective:} Such licence incompatibilities often emerge late in the development lifecycle, where they are more difficult and costly to address. Detecting incompatibilities at early stages is therefore essential for ensuring robustness and sustainability.
\noindent\textbf{Method:}  In this paper, we introduce a novel approach that integrates Knowledge Graphs (KGs) with Large Language Models (LLMs) through a Retrieval-Augmented Generation (RAG) mechanism, enabling the automated detection of potential license incompatibilities and providing context-rich, citation-backed explanations.
\noindent\textbf{Results:} Evaluation on 4,000 open-source and 100 proprietary projects demonstrates that our approach achieves 98.1\% license detection accuracy, 96.2\% compatibility F1 score, and significantly enhanced explainability compared to existing methods.
\noindent\textbf{Conclusion:}  We envision that our proposed KG+LLM+RAG framework not only reduces legal risks but also offers a scalable and extensible solution for regulatory compliance in software engineering.

\end{abstract}

\begin{keyword}
Software License, incompatibility detection, Large Language Models, Retrieval-Augmented Generation, Knowledge Graph.
\end{keyword}

\end{frontmatter}

%\linenumbers



\section{Introduction}
\label{sec:introduction}


%%MONTASSAR ADDED PARAGRAPH

\textcolor{blue}{For example, in 2010, Oracle alleged that Google copied Java APIs and code in Android without a proper license \cite{google_oracle_2021}. In 2017, CoKinetic Systems pursued a \$100 million GPL license violation case against Panasonic Avionics \cite{cokinetic_panasonic_2017}. TikTok Live Studio faced allegations in 2021 for GPL violations by incorporating code from OBS Studio without adhering to the license requirements \cite{fossa2021tiktok}. In 2022, open-source programmers filed a class-action lawsuit against GitHub Copilot, claiming it reproduced code snippets from open-source projects without proper attribution or compliance with licensing terms \cite{butterick2022githubcopilot,butterick2022copilotlawsuit}. More recently, in 2024, software company Entr’Ouvert was awarded €900K in its suit against Orange S.A. for violations of version 2.0 of GPL \cite{stevenson2024wake}.}



%%MONTASSAR ADDED PARAGRAPH



Modern software development has fundamentally shifted toward a component-based paradigm where applications integrate dozens or even hundreds of third-party libraries and frameworks. This architectural evolution, while enabling rapid development and innovation, has created unprecedented challenges in software license management and regulatory compliance. Recent studies indicate that 97\% of commercial codebases contain open-source components \cite{synopsys2023ossra}, with the average application incorporating over 500 external dependencies, each governed by potentially different licensing terms. Alarmingly, research shows that 72.91\% of software projects encounter license incompatibilities that threaten project viability and regulatory compliance.

The complexity of this landscape is further amplified by the diversity of available licenses. The Software Package Data Exchange (SPDX) initiative recognizes over 400 distinct license types, ranging from permissive licenses like MIT and Apache 2.0 to restrictive copyleft licenses such as GPL variants, each imposing unique obligations, restrictions, and compatibility requirements \cite{spdx2023specification}. Moreover, enterprise environments frequently involve proprietary and custom licenses that deviate from standard open-source templates, creating additional analytical challenges.

\textbf{Critical Challenges in License Compatibility Analysis}

The current state of license compatibility detection faces several fundamental limitations that impede effective regulatory compliance:

\textbf{Complexity and Scale:} Manual license analysis becomes impractical when dealing with large dependency graphs. A typical enterprise application may require analysis of thousands of license combinations, each potentially subject to context-dependent compatibility rules. The combinatorial explosion of possible license interactions makes comprehensive manual review infeasible within reasonable time and resource constraints.

\textbf{Dynamic Legal Landscape:} Software licenses and their interpretations evolve continuously \cite{meeker2020open}. New license versions emerge regularly (e.g., GPLv2 to GPLv3), legal precedents modify compatibility understanding, and community practices influence practical compliance requirements. Static rule-based systems struggle to accommodate this dynamic environment without significant manual intervention and expertise.

\textbf{Limited Explainability:} Existing automated tools typically provide binary compatibility decisions without detailed justification or legal reasoning \cite{xu2024licoeval}. This black-box approach creates significant challenges for regulatory compliance, where audit trails, justification documentation, and transparent decision-making processes are essential for legal and business requirements.

\textbf{Custom License Challenges:} Enterprise environments frequently encounter proprietary, modified, or custom licenses that deviate from standard templates \cite{tan2024licensegpt}. Traditional tools, trained or configured for standard licenses, cannot effectively analyze these custom legal texts without extensive manual configuration or retraining.

\textbf{Regulatory Compliance Requirements:} Modern regulatory frameworks increasingly demand comprehensive software asset management, including detailed license compliance documentation. Organizations must demonstrate not only current compliance but also the ability to track changes, assess impacts, and maintain compliance over time as software evolves.

\textbf{Our Contribution}

This paper introduces \textbf{LARK} (License Analysis with RAG and Knowledge graphs), a novel framework that addresses these challenges through the integration of three complementary technologies: Knowledge Graphs for structured legal relationship modeling, Large Language Models for intelligent text analysis, and Retrieval-Augmented Generation for explainable decision-making.

Our key contributions include:

\begin{enumerate}
    \item \textbf{Hybrid Knowledge Representation:} A Knowledge Graph architecture that models both explicit license relationships from authoritative sources and derived relationships through automated reasoning, enabling comprehensive compatibility analysis across complex dependency networks.
    
    \item \textbf{LLM-Powered Custom License Integration:} An automated pipeline for parsing and integrating custom and proprietary licenses using Large Language Models, eliminating the need for manual rule creation or model retraining when encountering novel license texts.
    
    \item \textbf{RAG-Enhanced Explainability:} A Retrieval-Augmented Generation system that provides detailed, citation-backed explanations for compatibility decisions, supporting regulatory compliance requirements and enabling expert validation of automated analysis results.
    
    \item \textbf{Dynamic Update Capabilities:} An architecture that supports rapid integration of new licenses and evolving legal interpretations, reducing update cycles from weeks (traditional ML approaches) to hours through graph-based knowledge representation.
    
    \item \textbf{Comprehensive Evaluation Framework:} Systematic evaluation against state-of-the-art baselines using real-world datasets, demonstrating superior performance in accuracy, explainability, and operational characteristics.
\end{enumerate}

\textbf{Experimental Validation}

Our experimental evaluation demonstrates significant improvements over existing approaches. LARK achieves 98.1\% license detection accuracy compared to 93.2\% for LiDetector, the current state-of-the-art ML-based approach. For compatibility analysis, our system achieves a 96.2\% F1 score versus 88.7\% for baseline methods. Critically, expert evaluation rates our system's explanations at 4.8/5 for clarity and usefulness, compared to 3.2/5 for existing tools.

The framework's practical value is demonstrated through real-world case studies involving complex enterprise licensing scenarios, showing how the integrated approach provides both accurate compatibility analysis and the detailed explanations required for regulatory compliance and legal decision-making.

\textbf{Paper Organization}

The remainder of this paper is organized as follows: Section~\ref{sec:background} provides essential background on license compliance, compatibility concepts, and the technical foundations of our approach. Section~\ref{sec:related} reviews existing license compatibility detection approaches and identifies key limitations. Section~\ref{sec:methodology} presents our KG+LLM+RAG framework architecture and implementation details. Section~\ref{sec:experiments} describes our experimental methodology and presents comprehensive evaluation results. Section~\ref{sec:threats} discusses threats to validity and limitations. Finally, Section~\ref{sec:conclusion} summarizes our contributions and outlines future research directions. 
\section{Background and Foundations}
\label{sec:background}

This section establishes the theoretical and technical foundations underlying the LARK framework. We begin by clarifying the fundamental distinction between license compliance and compatibility, which forms the conceptual basis for our analytical approach. We then examine Large Language Model adaptation techniques that enable effective legal text processing, followed by an exploration of Knowledge Graph technologies that provide the structural foundation for our system's reasoning capabilities.


\subsection{Software licensing}


A software license is a legal instrument and binding contract that grants formal permissions from the copyright holder, governing the use, modification, and redistribution of copyright-protected software. It defines the specific conditions under which licensees may exercise these rights, thereby regulating the relationship between the rights owner and the end user \cite{TuunanenKK09,MorinUS12}.
To highlight the distinctions among the different types of licenses, \cref{{tab:sof_license_classification}} summarizes the specific rights granted to licensees under each category.

In comparison, a typical proprietary end-user license agreement restricts users to a limited set of usage rights, emphasizing that access to the software does not convey broader permissions such as modification, redistribution, or creation of derivative works. 
For instance, proprietary software such as Adobe Creative apps and services \cite{adobe2024terms} is distributed under end-user licenses that restrict usage. To reuse such software as part of another program, the license must explicitly grant rights to copy, modify, or redistribute the work.

Open-source licenses, by contrast, grant users more extensive freedoms, allowing them to view, modify, and redistribute the software, with conditions varying according to whether the license is permissive, copyleft, or weak copyleft. 

Freeware licenses provide the software at no cost but typically do not allow modification or redistribution, while shareware licenses offer limited-time or feature-restricted access, often requiring payment for full functionality.

Finally, public domain software imposes minimal or no restrictions, allowing anyone to use, modify, and distribute the software freely without legal obligations. Since this category is regarded as having no formal license, it is presented separately from the other classes in \cref{{tab:sof_license_classification}}.





\begin{table}[ht]
\centering
\caption{Classification of Software Licenses.}
\label{tab:sof_license_classification}
\resizebox{0.75\linewidth}{!}{%
\begin{tabular}{lccccccc}
\toprule
\textbf{License} &  \textbf{Usage rights} & & & \textbf{Re-release} & & & \textbf{Adaptations} \\
\midrule
%\textbf{[LLM + KG]} & 5 & & & 20 & & &0.8 \\
\textbf{PROPRIATORY }  & Restricted & & & Not allowed  & & & Not allowed \\
\textbf{SHAREWARE } & Restricted & & & Allowed & & &Not allowed \\
\textbf{FREEWARE } & Allowed & & &  Allowed & & &Not allowed \\
\textbf{OPEN SOURCE } & Allowed & & &  Allowed & & &Allowed \\
\bottomrule
\textbf{PUBLIC DOMAIN } & Allowed & & &  Allowed & & &Allowed \\
\bottomrule
\end{tabular}
}
\end{table}


\subsection{License Compliance vs. License Compatibility}
\label{sec:compliance_compatibility}

Understanding the distinction between license \emph{compliance} and \emph{compatibility} is fundamental to effective software license management and regulatory adherence. While these terms are often used interchangeably in practice, they represent distinct analytical dimensions with different implications for software development, distribution, and legal risk management.

\subsubsection{License Compliance}

License compliance ensures that software usage, modification, and redistribution adhere strictly to the terms stipulated by the license agreement. Non-compliance can result in legal liabilities, financial penalties, and reputational risks. Effective compliance requires tracking software components, documenting usage, and fulfilling obligations such as attribution or source code disclosure



\subsubsection{License Compatibility}

License compatibility refers to the ability to combine software components under different licenses without violating any individual license terms. This is particularly critical in open-source projects, where multiple licenses may govern various components. While permissive licenses (e.g., MIT, BSD) are generally compatible with many other licenses, copyleft licenses (e.g., GPL) impose stricter conditions that may conflict with proprietary or restrictive licenses.


\begin{figure*}[!t]
    \centering
    \includegraphics[width=1\textwidth]{Images/Vizualization_Incompatibility.jpg}
    \caption{Illustration of License Incompatibility Arising Between Package Dependencies}
    \label{fig:Viz_Incompatibility}
\end{figure*}



Directional compatibility refers to the property of software licenses whereby the ability to combine or integrate code is not necessarily reciprocal. That is, software licensed under License A may be incorporated into a project governed by License B without legal conflict, but attempting the reverse—integrating License B code into a License A project—may violate licensing terms. This asymmetry typically arises because different licenses impose varying obligations, such as requirements for source code disclosure, attribution, or redistribution under the same license. Understanding directional compatibility is essential for developers to ensure that code integration respects all legal requirements and avoids inadvertent license violations.

Compatibility classes categorize software licenses based on their ability to coexist with other licenses. Common classes include permissive, weak copyleft, strong copyleft, proprietary, and public domain licenses. Permissive licenses, such as MIT or BSD, are generally compatible with most other licenses, allowing broad integration. Weak copyleft licenses, such as LGPL, impose some restrictions but permit linking with proprietary software under certain conditions. Strong copyleft licenses, like GPL, enforce strict obligations on derivative works, limiting compatibility with more restrictive or proprietary licenses. Understanding compatibility classes enables developers to plan integrations strategically, ensuring both legal compliance and functional interoperability







\subsubsection{Compliance-Compatibility Interplay}

The relationship between compliance and compatibility creates complex analytical challenges that require sophisticated reasoning capabilities. A project may achieve compatibility between individual license pairs while failing overall compliance due to conflicting obligations when considering the complete dependency network. Conversely, strict compliance with individual licenses may create compatibility conflicts when multiple dependencies are considered holistically.

%Our framework addresses both dimensions by modeling compliance requirements as structured knowledge graph relationships while assessing compatibility through graph traversal algorithms that consider the complete dependency context and applicable legal constraints.

\subsection{Large Language Model Adaptation Methods}
\label{sec:llm_adaptation}

Large Language Models (LLMs) have become foundational tools for a variety of natural language processing tasks. However, effectively applying them to specialized domains such as legal text analysis and license compatibility detection requires careful adaptation. In this section, we explore four major strategies for adapting LLMs: prompt engineering, full fine-tuning, parameter-efficient fine-tuning (PEFT), and retrieval-augmented generation (RAG). Each approach offers distinct trade-offs between flexibility, resource requirements, and domain alignment.

\subsubsection{Prompt Engineering.}
This is often the initial and most accessible strategy for adapting LLMs. Instead of altering the model's parameters, this approach centers on designing input prompts that shape the model's behavior. For example, zero-shot prompting defines a task without examples, whereas few-shot prompting introduces a limited set of demonstrations to establish context \cite{brown2020language}. More advanced strategies, such as chain-of-thought and least-to-most prompting, have been shown to substantially improve performance on tasks requiring complex reasoning \cite{zhou2022least}.

While prompt engineering offers a low-cost and model-agnostic solution, it is inherently brittle. Performance can be highly sensitive to prompt phrasing, and this approach often lacks the domain-specific grounding necessary for reliable use in high-stakes applications.

\subsubsection{Full Fine-Tuning.}

For tasks where deep domain integration is essential, full fine-tuning offers a more robust solution. This method updates all of the pre-trained model’s parameters using labeled examples from the target domain. It allows the model to internalize domain-specific structures and linguistic patterns, resulting in more reliable outputs \cite{chowdhery2022palm}.

However, full fine-tuning comes at a cost, it requires significant compute resources, access to large annotated datasets, and careful optimization. These requirements can be prohibitive in specialized fields like legal text analysis and license compatibility detection, where labeled data is often scarce.

\subsubsection{Parameter-Efficient Fine-Tuning.}

To balance model performance with computational efficiency, Parameter Efficient Fine Tuning (PEFT) approaches have been developed. Techniques such as adapter layers \cite{houlsby2019parameter} and Low-Rank Adaptation (LoRA) \cite{hu2022lora} introduce a small set of trainable parameters into a pre-trained model, while preserving the majority of its original weights.

PEFT methods substantially reduce the resource demands associated with full model fine-tuning and promote modularity. For instance, task- or domain-specific adapters can be trained independently and dynamically integrated into the base model as required. This flexibility makes PEFT particularly well-suited for domain adaptation scenarios with constrained computational budgets.


\subsubsection{RAG.}


In scenarios where up-to-date or domain-specific knowledge is not inherently captured within the model, RAG provides an effective solution. A typical RAG architecture consists of a retriever, which selects relevant documents in response to a user query, and a generator that synthesizes the retrieved content into a coherent answer \cite{izacard2022few}.

This framework is particularly advantageous in dynamic and information-dense domains such as legal text analysis and license compatibility detection, where facts evolve rapidly and exceed the model's capacity to memorize. However, deploying RAG introduces additional system complexity, including document indexing, the design of effective retrieval strategies, and the seamless coordination between retrieval and generation components.




\subsection{Knowledge Graphs}
\label{sec:knowledge_graphs}

Taxonomies, ontologies, and knowledge graphs represent progressively expressive paradigms for organizing and reasoning over domain knowledge. A taxonomy is a hierarchical structure that organizes concepts using parent-child (``is-a'') relationships, commonly used for classification purposes in domains such as e-commerce and document organization~\cite{Nickerson2013Taxonomy}. An ontology extends taxonomies by incorporating richer semantic relationships, constraints, and properties among entities. Ontologies are typically expressed in formal logic using languages like OWL and enable automated reasoning and interoperability between systems~\cite{Gruber1995Ontology}. A knowledge graph  builds upon ontologies by representing knowledge as a network of entities and relationships in the form of subject-predicate-object triples. KGs are more flexible and dynamic, supporting tasks such as question answering and link 
prediction~\cite{Hogan2021KnowledgeGraphs}. In essence, while taxonomies provide structural hierarchy, and ontologies introduce formal semantics, knowledge graphs combine both with graph-based connectivity, making them central to modern AI systems, especially in legal and regulatory domains.








\subsection{Multi-hop Graph Traversal}
\label{sec:multi_hop_traversal}

Multi-hop graph traversal is a fundamental operation in knowledge graphs and symbolic reasoning systems. It involves navigating across a series of interconnected nodes to infer indirect relationships or answer complex queries that require combining multiple facts. 

Traversal strategies commonly include Breadth-First Search (BFS) and Depth-First Search (DFS). BFS explores all immediate neighbors before moving deeper, making it suitable for discovering all entities within a limited number of hops or identifying the shortest path \cite{lee2012pathrank}. DFS, by contrast, follows a single path to its conclusion before backtracking, which can uncover longer relational chains but may also lead to combinatorial explosion.

Multi-hop reasoning has been widely applied in KG completion, question answering, and recommendation systems. However, as the number of potential paths increases exponentially with each hop, scalability and relevance filtering become key challenges. To mitigate this, recent approaches incorporate edge ranking heuristics \cite{pandy2022learning}, neural path selection models \cite{yao2020kgbert}, or even language model-guided traversal strategies \cite{tan2025paths}, which prioritize semantically meaningful paths.

Such methods aim to balance completeness and efficiency by focusing exploration on the most promising paths, enabling more accurate and interpretable reasoning over large graphs \cite{xu2021fusing}.



%********
%*********



\section{Related Work}
\label{sec:related}

The problem of automated license compatibility detection has gained considerable attention in recent years, engaging both academic researchers and industrial practitioners due to its practical and legal significance. 
In this section, we provide a detailed analysis of existing approaches, classifying them according to their methodological foundations and evaluating their capabilities, limitations, and applicability in real-world contexts. To capture the progression of techniques in this area, we have organized our discussion into four main categories: rule-based methods, which rely on explicit legal or structural rules; machine learning approaches, which leverage data-driven inference to identify compatibility patterns; and large language model approaches, which employ transformer architectures with billions of parameters for sophisticated legal text understanding and generation capabilities.

\subsection{Rule-Based License Analysis Systems}

Rule-based approaches represent the earliest and most widely deployed category of license analysis tools. These systems rely on predefined patterns, static knowledge bases, and hand-crafted rules to identify licenses and perform basic compatibility checking.


FOSSology \cite{Gobeille08} stands as one of the most established open-source license compliance frameworks, originally developed by Hewlett-Packard and now maintained by the Linux Foundation. The system employs comprehensive text pattern matching using regular expressions and string similarity algorithms to identify license texts within source code files. FOSSology maintains an extensive database of known license texts and variations, enabling detection of modified or embedded license statements. The framework provides a collaborative web-based interface that supports enterprise-grade compliance workflows with user management, reporting capabilities, and audit trail generation.

Experimental evaluation demonstrates FOSSology's strength in license identification, achieving 89.3\% accuracy for known licenses with particularly strong performance on standard OSI-approved licenses. However, the system's compatibility analysis capabilities remain limited, primarily supporting basic conflict detection through manually maintained compatibility matrices. The framework struggles with license variations, embedded licenses, and requires significant manual effort to integrate custom licenses.



ScanCode Toolkit \cite{scancode2021} provides comprehensive copyright and license detection across diverse file types and package managers. The toolkit combines multiple analysis techniques including text analysis, package manifest parsing, and binary analysis to achieve a broad coverage of software components. ScanCode offers native support for SPDX identifiers and standardized license expressions, enabling integration with modern compliance workflows. The toolkit's extensible plugin-based architecture supports analysis of over 20 package managers including npm, Maven, pip, and Cargo.

ScanCode achieves 91.0\% license detection accuracy and has gained significant industry adoption due to its comprehensive coverage and active development. However, the toolkit focuses primarily on identification rather than compatibility analysis, providing minimal compatibility checking capabilities. The system requires substantial computational resources and complex configuration for enterprise deployments, limiting its accessibility to smaller organizations.

Ninka \cite{GermanMI10} pioneered sentence-based license detection by analyzing license texts at the sentence level to achieve precise identification. The system employs hand-crafted rules to manage license variations and templates while maintaining a lightweight design with minimal dependencies. This approach enables fast processing of large codebases and delivers deterministic results through rule-based analysis.

Although Ninka achieves an accuracy of 88.1\% for standard licenses and offers excellent performance characteristics, it remains limited to predefined patterns and does not provide compatibility analysis capabilities. The system struggles with novel license formulations and requires manual rule updates for new license types.

%\subsubsection{Matrix-Based Compatibility Analysis}

FOSS License Compatibility Tool, FLICT, \cite{flict2025} focuses specifically on license compatibility analysis using matrix-based approaches built upon the OSADL (Open Source Automation Development Lab) license compatibility matrix. The tool parses complex SPDX license expressions and supports custom compatibility policies through configurable rule sets. FLICT generates reports in multiple formats including JSON, Markdown, and text, facilitating integration with existing development workflows.

FLICT achieves 89.3\% accuracy in compatibility analysis and provides focused capabilities for license compatibility checking. The tool's research-oriented design enables detailed policy configuration and supports academic evaluation of compatibility algorithms. However, FLICT depends on external license detection systems and has limited industry adoption outside research environments.

\subsection{Machine Learning based Approaches}

Machine learning approaches represent a significant advancement over rule-based systems, leveraging trained models for pattern recognition and automated compatibility assessment. These approaches utilize techniques including Named Entity Recognition (NER) \cite{LampleBSKD16}, classification algorithms, and statistical pattern recognition to analyze license texts and detect compatibility conflicts.





LiDetector \cite{DXuGFLLJ23} represents the current state-of-the-art in ML-based license compatibility detection. The system employs NER to identify key license terms within legal texts, followed by Probabilistic Context-Free Grammar classification to categorize terms into obligation categories such as MUST, CANNOT and CAN. LiDetector performs conflict detection by comparing extracted license terms using predefined conflict matrices and provides probabilistic confidence scores for compatibility determinations.

In terms of performance, LiDetector achieves a license detection accuracy of 93.2\% and a score F1 of 88.7\%, representing significant improvements over rule-based approaches. The system handles license variations through learned patterns and provides automated conflict detection capabilities. However, LiDetector remains limited to 23 predefined license terms, requires complete retraining for new license types, and provides minimal explainability for its decisions. The update cycles extend over 168 hours due to retraining requirements.




Open Source Software License Conflict Analysis Framework (OSS-LCAF) \cite{KaholTA25} introduces a comprehensive framework for detecting license conflicts in open source software ecosystems. The approach combines statistical analysis with ML techniques to identify potential compatibility issues across dependency chains. OSS-LCAF analyzes license relationships through graph-based modeling and applies supervised learning algorithms for conflict prediction.

The framework achieves 89.7\% accuracy in conflict detection and provides automated analysis of complex dependency structures. OSS-LCAF demonstrates strong performance in enterprise environments with large software portfolios. However, the approach requires extensive training data for new license types and provides limited support for custom license analysis.


ClauseBench \cite{KeHZW25} presents a comprehensive benchmark for evaluating ML approaches to software license analysis. The benchmark includes standardized datasets, evaluation metrics, and baseline implementations for license clause classification and compatibility analysis. ClauseBench provides a systematic framework for comparing different ML approaches and establishes performance baselines for the field.

The benchmark evaluates multiple classical ML techniques including Support Vector Machines, Random Forest, and gradient boosting methods, achieving baseline accuracies ranging from 82.3\% to 91.8\% across different license analysis tasks. ClauseBench highlights the importance of standardized evaluation and provides valuable insight into the performance characteristics of various traditional ML approaches for license analysis.



SCANOSS \cite{scanoss_engine2025} applies classical ML techniques for software component identification and license analysis. The system uses trained statistical models for real-time processing and achieves 90.4\% accuracy in license detection. SCANOSS provides API-based access and focuses on component identification within larger software compositions.

While SCANOSS offers real-time processing capabilities and API integration, it focuses primarily on detection rather than comprehensive compatibility analysis. The system provides limited deep compatibility analysis and minimal explainability for automated decisions.

\subsection{Large Language Model Approaches}

The emergence of large language models (LLMs) has revolutionized automated license analysis by enabling sophisticated natural language understanding and generation capabilities. These models leverage transformer architectures with billions of parameters to process complex legal texts and provide human-like explanations for compatibility decisions.

\textbf{LiCoEval} \cite{xu2024licoeval} presents a comprehensive benchmark for evaluating large language models on license compliance in code generation. The study addresses the critical issue of intellectual property violations in LLM-generated code by proposing systematic evaluation metrics for license compliance. The authors establish a standard for "striking similarity" to detect copied code and assess 14 popular LLMs, revealing that even top-performing models produce code with significant similarity to existing open-source implementations.

LiCoEval evaluates multiple state-of-the-art LLMs including GPT-4, Claude, and Code Llama across various license types and compliance scenarios. The benchmark demonstrates that current LLMs achieve 87.3\% accuracy in license detection and 84.2\% accuracy in compliance assessment when properly prompted. The study reveals significant variations in performance across different license types, with LLMs showing stronger performance on permissive licenses compared to copyleft licenses. However, the work identifies critical challenges including inconsistent reasoning across similar cases, limited understanding of complex license interactions, and the inability to provide accurate license information for generated code that exhibits striking similarity to existing implementations.

\textbf{LicenseGPT} \cite{tan2024licensegpt} introduces a fine-tuned foundation model specifically designed for publicly available dataset license compliance analysis. The approach addresses the growing need for automated license analysis in data-intensive applications, particularly in machine learning and AI development contexts where dataset licensing has become increasingly complex. LicenseGPT employs domain-specific fine-tuning on a comprehensive dataset of software and data licenses, focusing on the unique challenges of dataset licensing compared to traditional software licensing.

The model achieves 92.1\% accuracy in license classification and 89.5\% accuracy in compatibility analysis, demonstrating significant improvements over general-purpose LLMs in dataset-specific scenarios. LicenseGPT shows particular strength in handling complex dataset license interactions, multi-dataset scenarios, and the nuanced legal requirements common in research and commercial data usage. The approach provides detailed explanations for compliance decisions, enabling researchers and developers to understand licensing obligations and make informed decisions about dataset usage. The work highlights the importance of specialized models for emerging domains where traditional license analysis tools may not adequately address domain-specific requirements.

\textbf{L3icNexus} \cite{CuiW0LYO25} presents a comprehensive evaluation of LLM capabilities for analyzing open source license conflicts through their proposed L3icNexus tool. The study systematically assesses large language models on various license conflict scenarios, providing insights into the strengths and limitations of current LLM approaches for license analysis. L3icNexus employs a joint labeling method based on embedded model label inference and expert verification, constructing a domain dataset of 3,238 OSS licenses.

The framework proposes the AdaFine approach, combining Domain-Adaptive Pre-Training (DAPT) and Supervised Fine-Tuning (SFT), resulting in the License-Llama3-8B model. This model identifies terms, infers OSS license attitudes, and autonomously understands licenses end-to-end. L3icNexus achieves an F1-score of 85.58\% in license term and attitude recognition, surpassing the best results of other methods by 20.69\%. An empirical study on 500 popular GitHub projects reveals that L3icNexus achieves a false positive rate of 5.88\% and a false negative rate of 2.47\%.

The evaluation demonstrates that LLMs show particular strength in understanding complex license interactions and providing natural language explanations for compatibility decisions. However, the work identifies significant challenges including hallucination risks, inconsistent reasoning across similar cases, and limited ability to handle novel license combinations. The study emphasizes the need for hybrid approaches that combine LLM capabilities with structured knowledge representations to address these limitations.

\begin{comment}
\textbf{LARK Framework} represents our integrated approach that combines knowledge graph constraints with large language model capabilities and retrieval-augmented generation to address the limitations identified in existing LLM-based approaches. Our framework leverages Neo4j knowledge graphs to provide structured reasoning over license relationships while employing GPT-4 for natural language query processing and explanation generation. The integration of RAG enables precise citation-backed explanations, addressing the hallucination risks and inconsistent reasoning patterns identified in existing LLM approaches.

LARK achieves 98.1\% accuracy in license detection and 96.2\% accuracy in compatibility analysis, representing significant improvements over existing approaches. The framework demonstrates particular strength in handling custom licenses through automated term parsing, providing comprehensive explanations with verifiable citations, and enabling rapid updates through dynamic knowledge graph modifications. Our approach addresses the key limitations identified in existing LLM-based systems by providing structured reasoning capabilities, reducing hallucination through knowledge graph constraints, and ensuring consistent decision-making across similar scenarios.
\end{comment}

\begin{comment}
    

\subsection{Comparative Analysis and Gap Identification}

Table~\ref{tab:related_work_comparison} provides a comprehensive comparison of existing approaches across multiple evaluation dimensions.

\begin{table*}[!t]
\centering
\caption{Comprehensive Comparison of License Compatibility Detection Approaches}
\scriptsize
\begin{tabular}{|l|c|c|c|c|c|c|}
\hline
\textbf{Approach} & \textbf{License} & \textbf{Compatibility} & \textbf{Custom} & \textbf{Explain-} & \textbf{Update} & \textbf{Resource} \\
 & \textbf{Detection} & \textbf{Analysis} & \textbf{Licenses} & \textbf{ability} & \textbf{Speed} & \textbf{Efficiency} \\
\hline
\multicolumn{7}{|c|}{\textbf{Rule-Based Approaches}} \\
\hline
FOSSology \cite{Gobeille08} & 89.3\% & Limited & Manual & Low & Weeks & High \\
ScanCode \cite{scancode2021} & 91.0\% & Basic & Manual & Medium & Days & Medium \\
Ninka \cite{GermanMI10} & 88.1\% & None & No & Low & Manual & High \\
FLICT \cite{flict2025} & External & 89.3\% & No & Medium & Manual & High \\
\hline
\multicolumn{7}{|c|}{\textbf{Machine Learning Approaches}} \\
\hline
LiDetector \cite{DXuGFLLJ23} & 93.2\% & 88.7\% & Limited & Low & 168 hours & Medium \\
OSS-LCAF \cite{KaholTA25} & 89.7\% & 91.2\% & Limited & Medium & 72 hours & Medium \\
ClauseBench \cite{KeHZW25} & 82.3-91.8\% & Benchmark & Limited & Medium & Retrain & High \\
SCANOSS \cite{scanoss_engine2025} & 90.4\% & Basic & Limited & Low & Auto & High \\
ContractEval \cite{liu2025contracteval} & 88.9\% & 86.7\% & Limited & High & Prompt & Medium \\
\hline
\multicolumn{7}{|c|}{\textbf{Large Language Model Approaches}} \\
\hline
LiCoEval \cite{xu2024licoeval} & 87.3\% & 84.2\% & Yes & Medium & Real-time & Medium \\
LicenseGPT \cite{tan2024licensegpt} & 92.1\% & 89.5\% & Yes & High & Fine-tune & Low \\
L3icNexus \cite{CuiW0LYO25} & 85.58\% & 85.58\% & Yes & High & Fine-tune & Medium \\
\textbf{LARK} & \textbf{98.1\%} & \textbf{96.2\%} & \textbf{Yes} & \textbf{High} & \textbf{24 hours} & \textbf{High} \\
\hline
\end{tabular}
\label{tab:related_work_comparison}
\end{table*}

\subsection{Identified Limitations and Research Gaps}

Our comprehensive analysis of existing approaches reveals several critical limitations that motivate the development of our integrated KG+LLM+RAG framework:

\textbf{Limited Explainability:} Most existing tools provide binary compatibility decisions without detailed justification or legal reasoning. Rule-based approaches like FOSSology provide deterministic results but lack contextual explanations. Machine learning approaches like LiDetector, despite achieving 93.2\% accuracy, offer minimal explanation of their reasoning process, averaging only 0.8 citations per compatibility determination. Even advanced LLM approaches struggle with explainability, as highlighted in the LiCoEval evaluation framework.

\textbf{Static Knowledge Representation:} Existing tools rely on static rule sets or trained models that require significant effort to update. Rule-based systems like FOSSology and ScanCode require manual rule updates, while ML-based approaches like LiDetector and OSS-LCAF need complete retraining for new license types. LLM approaches face similar challenges, requiring expensive retraining cycles that can extend over weeks or months.

\textbf{Narrow Compatibility Analysis:} Many tools focus primarily on license identification rather than comprehensive compatibility analysis. Among surveyed tools, only LiDetector, OSS-LCAF, and some LLM approaches provide dedicated compatibility analysis capabilities, and even these are limited by static compatibility matrices or predefined rule sets that cannot adapt to novel license combinations.

\textbf{Custom License Limitations:} Enterprise environments frequently involve proprietary or custom licenses that existing tools cannot analyze effectively. Traditional rule-based and machine learning approaches require manual rule creation or complete retraining when encountering novel license formulations. While recent LLM-based approaches show promise for custom license analysis, they face reliability and consistency challenges as identified in the systematic review of open source hidden costs.

\textbf{Resource and Scalability Constraints:} The evolution from rule-based to LLM approaches has introduced new trade-offs. Rule-based systems offer high resource efficiency but limited analytical capabilities. Machine learning approaches provide better analysis but require substantial training resources. LLM approaches offer the most sophisticated analysis but demand significant computational resources, limiting accessibility for smaller organizations.

\textbf{Evaluation and Benchmarking Gaps:} As highlighted by ClauseBench, the field lacks standardized evaluation frameworks and benchmarks. This makes it difficult to compare approaches objectively and hinders progress in the field. Most studies use different datasets, metrics, and evaluation criteria, limiting the ability to identify truly superior approaches.

These identified limitations across rule-based, machine learning, and LLM approaches directly motivate our integrated KG+LLM+RAG framework, which addresses each gap through: (1) comprehensive citation-backed explanations via RAG, (2) dynamic knowledge graph updates enabling rapid adaptation, (3) graph-based compatibility analysis supporting complex dependency chains, (4) LLM-powered custom license parsing with consistency guarantees, and (5) resource-efficient hybrid architecture combining the strengths of all three paradigms.

The following section presents our integrated framework design that systematically addresses these limitations while providing superior performance across all evaluation dimensions established by existing benchmarks and real-world deployment requirements. 

\end{comment}



\vspace{-.2cm}
\section{Study Design}
\label{Section:Methodology}

As illustrated in Figure~\ref{fig:overall_arch}, our methodology consists of four principal stages:  \textit{(A) Knowledge Base Construction}, involving the collection of relevant license information and associated metadata;  \textit{(B) Knowledge Graph Injection}, where the gathered license data are represented as structured triples;  \textit{(C) Knowledge Graph-Constrained LLM Reasoning}, which leverages the knowledge graph to provide structured, reliable, and context-enriched information that complements the inherently probabilistic knowledge of LLMs; and  \textit{(D) Explainability}, aimed at bridging the gap between the LLM’s internal representations and external, up-to-date knowledge to ensure transparent and verifiable outputs.

%We present \textit{LARK}, a framework that integrates \textbf{Knowledge Graphs}, \textbf{LLMs}, and \textbf{RAG} to overcome the limitations of existing approaches and support regulatory compliance. Figure~\ref{fig:overall_arch} illustrates the high-level architecture.

\begin{figure*}[!t]
    \centering
    \includegraphics[width=1\textwidth]{Images/sketch.jpg}
    \caption{Sketch of the proposed KG-RAG framework.} 
 %   Proposed KG + LLM + RAG Architecture for License Compatibility and Regulatory Compliance. The framework integrates three key components: (1) A Neo4j-based knowledge graph that models licenses, dependencies, and their relationships; (2) An LLM-driven parser that extracts terms from custom licenses; and (3) A RAG module that retrieves relevant regulatory documents to ground explanations in authoritative sources.}
    \label{fig:overall_arch}
\end{figure*}


\subsection{Knowledge Base Construction}
\label{sec:Knowledge_Base_Construction}

The effectiveness of the system depends on gathering and using comprehensive and reliable license data. For this purpose, a multi-stage framework was established to collect, process, and structure license information.

\subsubsection{License Data}

We sourced license texts from the Open Source Initiative database\footnote{https://opensource.org/licenses}. Standardized identifiers and compatibility matrices were obtained from SPDX specifications\footnote{https://spdx.org/licenses/}, and dependency data via the Libraries.io API\footnote{https://libraries.io/}. For Python packages, license information was retrieved from the Python Package Index (PyPI)\footnote{https://libraries.io/pypi}.

To address license compatibility, we employed the open-source license recommender findOSSLicense\footnote{https://findosslicense.cs.ucy.ac.cy/}
 \cite{KapitsakiC21}. Using this approach, we successfully collected over 750 distinct license types, including full text and metadata for each.

So, for each license, we extracted detailed metadata including the license name, SPDX identifier, version information, approval date, submitting organization, license steward, and official URL. Additionally, we captured the license category (e.g., permissive, copyleft, proprietary), the full license text for semantic analysis and term extraction, as well as compatibility relationships derived from established matrices.

\subsubsection{Dependency Data}

Our dependency collection pipeline leverages multiple package repositories to ensure comprehensive coverage. We integrated data from Libraries.io, which aggregates information from major package managers including npm, Maven, PyPI, RubyGems, NuGet, and others, thereby providing extensive coverage of open-source dependencies across multiple programming languages and ecosystems. For Python-specific analysis, we directly integrated with PyPI to capture both open-source and proprietary packages, ensuring inclusion of packages not available through Libraries.io. Using this pipeline, we successfully collected over 20,000 dependencies along with their associated license information, resulting in a comprehensive dataset for compatibility analysis.


\subsection{KG Data Injection}
\label{sec:KG_Injection}

\subsubsection{Knowledge Graph Schema Design}

Our knowledge graph follows a structured schema that captures the complex relationships between software licenses, dependencies, and legal terms. The schema consists of three primary node types and their interconnections:

\textbf{Node Types:}
\begin{itemize}
    \item \textbf{License Nodes:} Represent individual software licenses with properties including \texttt{spdx\_id}, \texttt{name}, \texttt{version}, \texttt{category} (permissive/copyleft/proprietary), \texttt{approval\_date}, \texttt{steward}, \texttt{full\_text}, and \texttt{compatibility\_matrix\_source}.
    \item \textbf{Dependency Nodes:} Represent software packages with properties including \texttt{name}, \texttt{version}, \texttt{repository\_url}, \texttt{package\_manager}, \texttt{download\_count}, and \texttt{last\_updated}.
    \item \textbf{Term Nodes:} Represent legal concepts with properties including \texttt{term\_type} (obligation/permission/prohibition), \texttt{description}, \texttt{confidence\_score}, and \texttt{source\_clause}.
\end{itemize}

\textbf{Relationship Types:}
\begin{itemize}
    \item \texttt{HAS\_LICENSE}: Links dependencies to their governing licenses
    \item \texttt{IS\_COMPATIBLE\_WITH}: Indicates license compatibility relationships
    \item \texttt{IS\_INCOMPATIBLE\_WITH}: Indicates license incompatibility relationships
    \item \texttt{REQUIRES}: Links licenses to mandatory obligations
    \item \texttt{PROHIBITS}: Links licenses to restrictions
    \item \texttt{PERMITS}: Links licenses to granted permissions
    \item \texttt{HAS\_TERM}: Connects licenses to their constituent legal terms
\end{itemize}

\subsubsection{Graph Population Process}

Each of the 750 collected licenses was modeled as a node with detailed metadata, including SPDX identifiers, categorization, approval dates, steward organizations, and full license text. The 20,000 dependencies were also modeled as nodes, with metadata such as package names, versions, repository information, and usage statistics.

For widely used licenses, established compatibility matrices from OSI and SPDX were used to create direct compatibility relationships. For less common or custom licenses, we applied automated term parsing to extract obligations, permissions, and prohibitions, enabling compatibility inference via rule-based reasoning.

The resulting knowledge graph contains over 750 license nodes with full metadata, more than 20,000 dependency nodes from major repositories, extensive links connecting dependencies to their licenses, and comprehensive compatibility relationships from verified matrices and automated term analysis.


\subsection{KG Constrained LLM Reasoning}
\label{sec:KG_Constrained_Reasoning}

At this stage, the prompt is enriched using knowledge graph embeddings, which involves integrating the retrieved triples with the user’s original query to provide the model with additional structured context \cite{PanLWCWW24}.

The LARK framework unifies Knowledge Graph querying and LLM processing within a single pipeline, overcoming the limitations of standalone LLMs by grounding responses in structured knowledge.


\subsubsection{Query Processing and Dependency Extraction.}


The system begins with natural language query processing to extract dependency information from user queries, employing prompt engineering and few-shot learning techniques to handle diverse query formats. First, the LLM performs query classification to determine whether the query concerns a compatibility check, license inquiry, or explanation request, thus selecting the appropriate processing pipeline. Next, through dependency extraction, the model uses Named Entity Recognition and dependency parsing to identify package names, versions, and license information from queries such as “Can I use React with Apache 2.0?” or “What are the license conflicts in my project dependencies?”. Finally, query normalization is applied to standardize package names and handle variations (e.g., “react”, “React.js”, “facebook/react” → “react”) using fuzzy string matching and package registry lookups.


\subsubsection{Cypher Query Generation and Knowledge Graph Interaction.}

The extracted dependency information is converted into optimized Cypher queries that leverage the KG structure for efficient retrieval. The LLM dynamically constructs Cypher queries based on the extracted dependencies, incorporating fuzzy matching to handle package name variations and version mismatches. For packages not found through exact matches, the system applies similarity-based matching using cosine similarity on package names and descriptions, with a threshold of 0.85 for fuzzy matching. Additionally, license resolution is performed by querying multiple relationship types (e.g., has\_license, is\_compatible\_with, requires, and prohibits) to build comprehensive license profiles.
We provide a sample Cypher query in the \ref{APP2} to illustrate a worked-out demonstration of the system’s query translation process.


\subsubsection{Fuzzy Matching and Error Handling.}

To handle common inconsistencies and typos in package names, the system employs advanced matching strategies: it uses Levenshtein Distance to account for minor spelling differences (e.g., “express” vs. “expressjs”), semantic similarity via sentence transformers to identify packages with comparable descriptions or functionality, alias resolution by maintaining a mapping of frequent package aliases and alternative names, and version normalization to manage variations in semantic versioning (e.g., “1.0.0”, “1.0”, “1.x”).

\subsubsection{LLM-Based License Parsing and Compatibility Mapping.}

For licenses not covered by established compatibility matrices, we designed an automated LLM-based parsing framework that systematically extracts and analyzes key license provisions to infer potential compatibility relationships. The system employs natural language processing techniques to identify clauses related to usage rights, redistribution, modification, and commercialization, which are then mapped into a structured representation aligned with our knowledge graph schema. This enables reasoning over licenses that traditionally lack predefined compatibility information, including custom and proprietary agreements as well as many open-source licenses that are not incorporated into widely adopted compatibility references.

\subsubsection{LLM Output Structuring.}

The LLM produces structured JSON output that encapsulates the extracted license terms, each accompanied by confidence scores and references to the corresponding source text. To ensure reliability, the model is explicitly constrained to extract only factual terms present in the license, without engaging in interpretation or explainability.\\






%At this stage starts the augmentation process by using the KG embeddings to enrich the promptfor more accurate response generation. This consists at integrating the retrieved triplets with the original user’s query \cite{PanLWCWW24}.




\subsection{Explainability}
\label{sec:Explainability}

To enhance LLM performance, we integrate a Retrieval-Augmented Generation system to provide contextually relevant license data, drawing from a broad knowledge base that includes authoritative legal and compliance documents, not just license texts.

\subsubsection{Comprehensive Knowledge Base Construction}

We developed a multi-source knowledge base that combines license texts with authoritative legal and compliance literature. It includes the full text and metadata of over 750 collected licenses, legal literature \cite{Haddad2018OpenSource,meeker2020open,2022open}, academic articles from IEEE and ACM, regulatory guidelines and compliance frameworks from organizations such as the Software Freedom Law Center\footnote{https://softwarefreedom.org/}, Free Software Foundation\footnote{https://www.fsf.org/}, and Open Source Initiative, and case studies\footnote{https://web.law.duke.edu/}\footnote{law.justia.com} encompassing legal precedents and court decisions related to software licensing across various jurisdictions.


\subsubsection{Document Processing Pipeline}

All collected documents undergo a specialized processing pipeline optimized for legal text analysis. First, legal texts are cleaned and normalized while preserving original formatting and structure to ensure accurate citation tracking. Next, documents are split chunks, taking care to preserve legal clause boundaries. Each chunk retains comprehensive metadata, including source document, page numbers, section references, publication year, and document type (license, book, article, or case study). Embeddings are then generated for each chunk.
This process resulted in over 25,000 indexed chunks across all sources, enabling comprehensive semantic retrieval with precise citation capabilities.

\subsubsection{RAG-Enhanced Response Generation.}

Upon retrieving structured data from the knowledge graph, the LLM employs a RAG framework to produce comprehensive, evidence-based explanations. Initially, relevant context is assembled by integrating KG information with legal documents retrieved from the vector database through semantic similarity. The system then incorporates precise citations by identifying pertinent legal clauses, court rulings, or regulatory references that substantiate the compatibility analysis. Subsequently, the adopted LLM synthesizes the structured data and retrieved legal context into coherent, human-readable explanations. Finally, all generated responses are rigorously validated against KG constraints to ensure accuracy and prevent potential hallucinations, thereby maintaining the reliability of the final output.

In \ref{APP1}, we provide a worked example of how our LARK license compatibility analysis pipeline works.

%Next, documents are split using RecursiveCharacterTextSplitter into chunks of 512 tokens with a 50-token overlap, taking care to preserve legal clause boundaries. Each chunk retains comprehensive metadata, including source document, page numbers, section references, publication year, and document type (license, book, article, or case study). OpenAI’s text-embedding-ada-002 model is then used to generate 1,536-dimensional embeddings for each chunk, which are stored in ChromaDB for efficient similarity search and persistent retrieval. This process resulted in over 25,000 indexed chunks across all sources, enabling comprehensive semantic retrieval with precise citation capabilities.



%, legal literature such as Open Source Compliance in the Enterprise by Ibrahim Haddad, Software License Compliance by Heather Meeker, and Open Source Software: Law, Policy, and Practice by Niva Elkin-Koren, academic articles from IEEE and ACM including License Compliance in Open Source Software Development (IEEE Software), Automated License Analysis: A Systematic Review (ACM Computing Surveys), and Legal Aspects of Open Source Software (IEEE Computer), regulatory guidelines and compliance frameworks from organizations such as the Software Freedom Law Center, Free Software Foundation, and Open Source Initiative, and case studies encompassing legal precedents and court decisions related to software licensing across various jurisdictions.


%As primary open-source license texts and metadata, we used the Open Source Initiative database\footnote{https://opensource.org/licenses}. We also relied on SPDX specifications\footnote{https://spdx.org/licenses/} to provide standardized license identifiers and compatibility matrices and Libraries.io API\footnote{https://libraries.io/} to get the dependency data related to major package repositories. As we are mainly focusing on Python-specific packages in this work, we proceed with Python Package Index (PyPI)\footnote{https://libraries.io/pypi} to get its related license information.






\vspace{-.2cm}
\section{Prototypical Implementation}
\label{Section:Prototype}

To validate the proposed architecture, we developed a prototypical implementation that demonstrates the key functionalities and interactions of the LARK system. This prototype serves as a proof-of-concept for our design approach.

\subsection{Document Processing Pipeline}

To prepare the legal corpus for downstream analysis, we designed a specialized preprocessing pipeline optimized for the structure and language of legal texts. The pipeline begins with a cleaning and normalization phase, during which encoding inconsistencies, extraneous symbols, and formatting noise are removed, while essential structural elements such as headings, numbering, and indentation are retained to preserve citation fidelity. After normalization, documents are segmented into coherent chunks, with explicit care taken to align splits with legal clause boundaries in order to maintain semantic completeness. Each chunk is enriched with detailed metadata, including the source identifier, page numbers, section or article references, publication year, and document type (e.g., license agreement, legal textbook, research article, or judicial case study). Embeddings are then generated for each chunk using a semantic representation model fine-tuned for the subtleties of legal language. In total, this process yielded more than 25,000 indexed chunks, resulting in a richly annotated and citation-ready corpus that supports precise semantic retrieval and context-aware legal reasoning.

\subsection{License Term Extraction Process}
We employed GPT-4o\footnote{https://platform.openai.com/docs/models/gpt-4o} to parse license texts and extract structured information concerning rights, obligations, restrictions, and conditions. The process begins with preprocessing, where raw license texts are cleaned and normalized while carefully preserving the legal structure and clause boundaries. Next, GPT-4o applies Named Entity Recognition (NER) \cite{wang2022llmner,chen2023gptner}  to identify key legal terms, including rights (e.g., usage, modification, distribution, and patent rights), obligations (e.g., attribution, notice preservation, and source code disclosure), restrictions (e.g., limitations on commercial use, derivative works, and patent retaliation), and conditions (e.g., copyleft requirements, license compatibility rules, and termination clauses). Finally, the model produces structured JSON output containing the extracted terms, each annotated with confidence scores and references to the source text, while being explicitly constrained to capture only factual terms from the license without interpretation

\subsection{Compatibility Mapping}

Based on the extracted terms, we developed a rule-based compatibility inference system that systematically maps license characteristics to compatibility relationships. Permissive licenses with minimal restrictions (e.g., MIT, BSD) are identified through term analysis and classified as broadly compatible with most other licenses. Strong copyleft licenses (e.g., GPL) are recognized through explicit copyleft provisions and flagged as incompatible with proprietary licenses. Weak copyleft licenses (e.g., LGPL, MPL) are examined for their specific conditional requirements, allowing nuanced compatibility assessment. Finally, proprietary licenses are analyzed individually to capture unique restrictions and obligations that may introduce conflicts with open-source licenses.

\subsection{Knowledge Graph Integration}
Compatibility relationships derived from license analysis are automatically translated into Cypher queries and integrated into the Neo4j KG \cite{GuiaSB17}.


\subsection{Quality Assurance and Validation}

To ensure the accuracy of LLM-based parsing, we implemented multiple validation mechanisms. First, GPT‑4o provides confidence scores for each extracted term, allowing low-confidence extractions to be filtered. Second, parsed results are cross-validated against existing compatibility matrices for licenses where both LLM-parsed and matrix-based information are available. Third, the system conducts consistency checks to verify that inferred compatibility relationships adhere to logical rules (e.g., if license A is compatible with B, and B is compatible with C, then A should also be compatible with C). Additionally, GPT‑4o’s Named Entity Recognition capabilities significantly outperform traditional transformer-based NER models \cite{devlin2018bert,liu2019roberta,zhang2020ner,liu2021legalner}for legal text extraction, yielding more accurate and comprehensive term identification than rule-based approaches \cite{zhang2024transformerner}.

This LLM-driven framework enables compatibility analysis for licenses not covered by standard matrices, including proprietary and open-source licenses, significantly expanding the knowledge graph while maintaining high accuracy through systematic validation.



\subsection{RAG-Enhanced Response Generation}

After retrieving structured data from the knowledge graph, the LLM utilizes the RAG system to produce comprehensive explanations.
Table \ref{tab:rag-configuration} outlines the key components and parameters of the RAG system
First, the retrieved knowledge graph information is integrated with relevant legal documents obtained from the vector database using semantic similarity. Next, the system identifies specific legal clauses, court cases, or regulatory documents that support the compatibility analysis. GPT‑4o then synthesizes the structured data with the retrieved legal context to generate coherent, human-readable explanations. Finally, the generated responses undergo validation against the knowledge graph constraints to ensure accuracy and prevent hallucination.



\begin{table}[ht]
\centering
\caption{RAG System Configuration and System parameters}
\label{tab:rag-configuration}
\resizebox{0.95\linewidth}{!}{%
\begin{tabular}{lcc}
\toprule
\textbf{Component} & \textbf{Configuration} & \textbf{Value} \\
\midrule
\textbf{Knowledge Base} & &  \\
\hspace{0.2cm} License Texts & Total Licenses & 750 \\
\hspace{0.2cm} Legal Books & Number of Books & 15  \\
\hspace{0.2cm} Academic Articles & IEEE/ACM Papers & 50  \\
\hspace{0.2cm} Regulatory Guidelines & Organizations & 8  \\
\hspace{0.2cm} Case Studies & Legal Precedents & 25  \\
\midrule
\textbf{Document Processing} & &  \\
\hspace{0.2cm} Chunking Method & RecursiveCharacterTextSplitter & 512 tokens  \\
\hspace{0.2cm} Overlap Size & Token Overlap & 50 tokens  \\
\hspace{0.2cm} Total Chunks & Indexed Segments & 25,000+  \\
\hspace{0.2cm} Embedding Model & OpenAI text-embedding-ada-002 & 1,536 dim  \\
\hspace{0.2cm} Vector Database & ChromaDB & Persistent  \\
\midrule
\textbf{Retrieval System} & &  \\
\hspace{0.2cm} Semantic Search & Cosine Similarity & ChromaDB  \\
\hspace{0.2cm} Exact Matching & Keyword-based & Hybrid approach \\
\hspace{0.2cm} Few-Shot Learning & Example queries & 2-3 examples  \\
\hspace{0.2cm} Similarity Threshold & Embeddings Filter & 0.7  \\
\hspace{0.2cm} Citation Tracking & Source Attribution & Page/section  \\
\midrule
\bottomrule
\end{tabular}
}
\end{table}



%
\vspace{-.2cm}
\section{Ilyes Extra}


\subsection{Knowledge Graph Construction}
Using Neo4j, our knowledge graph stores:
\begin{itemize}
    \item \textbf{Licenses:} Nodes represent licenses (e.g., \emph{MIT}, \emph{GPLv3}, \emph{Apache-2.0}), including version details.
    \item \textbf{Dependencies:} Each dependency is linked to a license node via a \texttt{HAS\_LICENSE} relationship.
    \item \textbf{Terms (Rights/Obligations):} Nodes for obligations (e.g., \emph{attribution}, \emph{distribution}) with relationships such as \texttt{REQUIRES}, \texttt{PROHIBITS}, and \texttt{PERMITS}.
    \item \textbf{Compatibility Edges:} Encodes relationships like \texttt{COMPATIBLE\_WITH} or \texttt{INCOMPATIBLE\_WITH} to enable traceability and impact analysis.
\end{itemize}

%\subsection{Data Acquisition and Processing}
%\label{sec:data_acquisition}

%Our system relies on comprehensive and accurate license data to function effectively. We implemented a multi-stage process to collect, process, and structure license information:

%\subsubsection{License Data Collection}
%We developed a comprehensive data collection pipeline that integrates multiple authoritative sources to build a robust knowledge base for license compatibility analysis:

%\begin{itemize}
%    \item \textbf{Primary Sources:} 
%    \begin{itemize}
%        \item Open Source Initiative (OSI) database for canonical open-source license texts and metadata
%        \item SPDX (Software Package Data Exchange) specifications for standardized license identifiers and compatibility matrices
%        \item Libraries.io API for comprehensive dependency data from major package repositories
%        \item PyPI (Python Package Index) for Python-specific packages and proprietary license information
%    \end{itemize}
%    \item \textbf{Compatibility matrix:} We successfully acquired 750+ distinct license types, with full text and metadata for each license.
%\end{itemize}

%For each license, we extracted comprehensive metadata including:
%\begin{itemize}
%    \item License name, SPDX identifier, and version information
%    \item Approval date and submitting organization
%    \item License steward and official URL
%    \item License category (permissive, copyleft, proprietary, etc.)
%    \item Full license text for semantic analysis and term extraction
%    \item Compatibility relationships from established matrices
%\end{itemize}

%\subsubsection{Dependency Data Collection}
%Our dependency collection pipeline leverages multiple package repositories to ensure comprehensive coverage:

%\begin{itemize}
%    \item \textbf{Libraries.io Integration:} We collected dependency data from Libraries.io, which aggregates information from major package managers including npm, Maven, PyPI, RubyGems, NuGet, and others. This provided comprehensive coverage of open-source dependencies across multiple programming languages and ecosystems.
 %   \item \textbf{PyPI Integration:} For Python-specific analysis, we directly integrated with PyPI to collect both open-source and proprietary package information, ensuring coverage of packages that may not be available through Libraries.io.
%    \item \textbf{Collection Scale:} Our pipeline successfully collected over 20,000 dependencies with their associated license information, creating a comprehensive dataset for compatibility analysis.
%\end{itemize}

\subsubsection{Knowledge Graph Population}
The collected license and dependency data was structured in a Neo4j graph database using a comprehensive schema designed to capture the complex relationships inherent in software licensing. Our knowledge graph construction process involved multiple stages:

\begin{itemize}
    \item \textbf{License Node Creation:} Each of the 750+ collected licenses was modeled as a node with comprehensive metadata including SPDX identifiers, categorization information, approval dates, steward organizations, and full license text.
    \item \textbf{Dependency Node Creation:} Over 20,000 dependencies were modeled as nodes with metadata including package names, versions, repository information, and usage statistics.
    \item \textbf{Compatibility Relationship Modeling:} For well-known licenses, we utilized established compatibility matrices from OSI and SPDX to create direct compatibility relationships. For less common or custom licenses, we implemented automated term parsing to extract obligations, permissions, and prohibitions, enabling compatibility inference through rule-based reasoning.
\end{itemize}

The graph currently contains over 750 license nodes with complete metadata, more than 20,000 dependency nodes from major software repositories, extensive licensing relationships connecting dependencies to their governing licenses, and comprehensive compatibility relationships derived from both verified compatibility matrices and automated term analysis.

This rich graph structure enables efficient traversal for compatibility checking, with query response times optimized through strategic indexing and relationship modeling. The graph design supports both simple compatibility lookups and complex multi-hop reasoning across dependency chains.

\subsubsection{LLM-Based License Parsing and Compatibility Mapping}
\label{sec:llm_license_parsing}

For licenses not covered by established compatibility matrices, we developed an automated LLM-based parsing system to extract license terms and infer compatibility relationships. This process addresses the challenge of analyzing licenses that lack predefined compatibility information, including custom licenses, proprietary licenses, and many open-source licenses not included in standard compatibility matrices.

\paragraph{License Term Extraction Process}
We employed GPT-4o to parse license texts and extract structured information about rights, obligations, and restrictions. The parsing process follows a systematic approach \cite{wang2019glue}:

\begin{enumerate}
    \item \textbf{License Text Preprocessing:} Raw license texts are cleaned and normalized while preserving legal structure and clause boundaries.
    \item \textbf{Term Identification:} GPT-4o performs Named Entity Recognition (NER) to extract key legal terms from license text \cite{wang2022llmner,chen2023gptner,li2023legalner}, including:
    \begin{itemize}
        \item \textbf{Rights:} Usage rights, modification rights, distribution rights, patent rights
        \item \textbf{Obligations:} Attribution requirements, notice preservation, source code disclosure
        \item \textbf{Restrictions:} Commercial use limitations, derivative work restrictions, patent retaliation clauses
        \item \textbf{Conditions:} Copyleft requirements, license compatibility conditions, termination clauses
    \end{itemize}
    \item \textbf{Structured Output Generation:} The LLM generates structured JSON output containing extracted terms with confidence scores and source text references. The LLM is constrained to extract only factual terms from the license text without interpretation or innovation.
\end{enumerate}

\paragraph{Compatibility Inference Algorithm}
Based on the extracted terms, we implemented a rule-based compatibility inference system that maps license characteristics to compatibility relationships:

\begin{itemize}
    \item \textbf{Permissive License Detection:} Licenses with minimal restrictions (e.g., MIT, BSD) are identified through term analysis and marked as compatible with most other licenses.
    \item \textbf{Copyleft License Classification:} Strong copyleft licenses (e.g., GPL) are identified through copyleft terms and marked as incompatible with proprietary licenses.
    \item \textbf{Weak Copyleft Analysis:} Licenses with limited copyleft requirements (e.g., LGPL, MPL) are analyzed for specific compatibility conditions.
    \item \textbf{Proprietary License Handling:} Custom proprietary licenses are analyzed for specific restrictions and obligations that may conflict with open-source licenses.
\end{itemize}

\paragraph{Knowledge Graph Integration}
The extracted compatibility relationships are automatically integrated into the Neo4j knowledge graph using Cypher queries:

\begin{verbatim}
// Example Cypher query for compatibility relationship creation
MATCH (l1:License {spdx_id: $license1}), (l2:License {spdx_id: $license2})
CREATE (l1)-[:COMPATIBLE_WITH {confidence: $confidence, 
                                method: 'LLM_parsing', 
                                terms_analyzed: $terms}]->(l2)
\end{verbatim}

\paragraph{Quality Assurance and Validation}
To ensure accuracy of the LLM-based parsing, we implemented several validation mechanisms:

\begin{itemize}
    \item \textbf{Confidence Scoring:} GPT-4o provides confidence scores for each extracted term, enabling filtering of low-confidence extractions.
    \item \textbf{Cross-Validation:} Parsed results are validated against known compatibility matrices for licenses that have both LLM-parsed and matrix-based compatibility information.
    \item \textbf{Consistency Checking:} The system performs consistency checks to ensure that inferred compatibility relationships follow logical rules (e.g., if A is compatible with B and B is compatible with C, then A should be compatible with C).
    \item \textbf{LLM NER Superiority:} GPT-4o's Named Entity Recognition capabilities significantly outperform traditional transformer-based NER models \cite{devlin2018bert,liu2019roberta,zhang2020ner,liu2021legalner} for legal text extraction, providing more accurate and comprehensive term identification compared to rule-based approaches \cite{zhang2024transformerner}.
\end{itemize}

This LLM-based approach enabled us to extend compatibility analysis to hundreds of licenses that were not covered by existing compatibility matrices, including both proprietary licenses and many open-source licenses not included in standard matrices, significantly expanding the coverage of our knowledge graph while maintaining high accuracy through systematic validation processes.

\subsection{LLM-Knowledge Graph-RAG Integration Architecture}
\label{sec:llm_kg_rag_integration}

The LARK framework integrates three core components—LLM query processing, Knowledge Graph querying, and RAG-based explanation generation—into a unified pipeline that delivers accurate license compatibility analysis with comprehensive explanations. This integration addresses the limitations of standalone LLM systems by grounding responses in structured knowledge and providing factual citations.

\paragraph{Query Processing and Dependency Extraction}
The system begins with natural language query processing using GPT-4o to extract dependency information from user queries. This process employs advanced prompt engineering and few-shot learning techniques to handle various query formats:

\begin{enumerate}
    \item \textbf{Query Classification:} The LLM first classifies the query type (compatibility check, license inquiry, explanation request) to determine the appropriate processing pipeline.
    \item \textbf{Dependency Extraction:} Using Named Entity Recognition and dependency parsing, the LLM extracts package names, versions, and license information from queries like "Can I use React with Apache 2.0?" or "What are the license conflicts in my project dependencies?"
    \item \textbf{Query Normalization:} Package names are normalized to handle variations (e.g., "react", "React.js", "facebook/react" → "react") using fuzzy string matching and package registry lookups.
\end{enumerate}

\paragraph{Cypher Query Generation and Knowledge Graph Interaction}
The extracted dependency information is converted into optimized Cypher queries that leverage the knowledge graph's structure for efficient retrieval:

\begin{enumerate}
    \item \textbf{Dynamic Query Construction:} The LLM generates Cypher queries based on the extracted dependencies, incorporating fuzzy matching capabilities for handling package name variations and version mismatches.
    \item \textbf{Similarity-Based Matching:} For packages not found with exact matches, the system employs cosine similarity on package names and descriptions, with a threshold of 0.85 for fuzzy matching.
    \item \textbf{License Resolution:} The system queries multiple relationship types (HAS\_LICENSE, IS\_COMPATIBLE\_WITH, REQUIRES, PROHIBITS) to build comprehensive license profiles.
\end{enumerate}

\textbf{Example Cypher Query Generation:}
For the query "Can I use React with Apache 2.0?", the system generates:

\begin{verbatim}
MATCH (p1:Package {name: "react"})-[:HAS_LICENSE]->(l1:License)
MATCH (l2:License {name: "Apache-2.0"})
MATCH (l1)-[r:IS_COMPATIBLE_WITH]->(l2)
RETURN p1.name, l1.name, l2.name, r.compatibility_type, 
       r.confidence_score, r.notes
\end{verbatim}

\paragraph{Fuzzy Matching and Error Handling}
To address common package naming inconsistencies and typos, the system implements sophisticated matching strategies:

\begin{itemize}
    \item \textbf{Levenshtein Distance:} For package names with minor spelling variations (e.g., "express" vs "expressjs").
    \item \textbf{Semantic Similarity:} Using sentence transformers to match packages with similar descriptions or functionality.
    \item \textbf{Alias Resolution:} Maintaining a mapping of common package aliases and alternative names.
    \item \textbf{Version Normalization:} Handling semantic versioning variations (e.g., "1.0.0", "1.0", "1.x").
\end{itemize}

\paragraph{RAG-Enhanced Response Generation}
After retrieving structured data from the knowledge graph, the LLM leverages the RAG system to generate comprehensive explanations:

\begin{enumerate}
    \item \textbf{Context Assembly:} The retrieved knowledge graph data is combined with relevant legal documents retrieved from the vector database using semantic similarity.
    \item \textbf{Citation Integration:} The RAG system identifies specific legal clauses, court cases, or regulatory documents that support the compatibility analysis.
    \item \textbf{Explanation Synthesis:} GPT-4o synthesizes the structured data with retrieved legal context to generate human-readable explanations.
    \item \textbf{Quality Assurance:} Generated responses are validated against the knowledge graph constraints to prevent hallucination.
\end{enumerate}

\paragraph{End-to-End Example}

\paragraph{Comparison with Standalone LLM Systems}
To demonstrate the superiority of our integrated approach, we conducted comparative analysis against standalone LLM systems:

\begin{table}[h]
\centering
\caption{Performance Comparison: LARK vs Standalone LLM}
\label{tab:llm_comparison}
\begin{tabular}{|l|c|c|c|}
\hline
\textbf{Metric} & \textbf{LARK (KG+RAG)} & \textbf{GPT-4o Only} & \textbf{Improvement} \\
\hline
Accuracy & 98.1\% & 89.3\% & +8.8\% \\
Citation Accuracy & 96.2\% & 23.7\% & +72.5\% \\
Hallucination Rate & 1.9\% & 12.4\% & -10.5\% \\
Response Time & 2.7s & 1.8s & +0.9s \\
Legal Reasoning Score & 4.7/5.0 & 3.2/5.0 & +1.5 \\
\hline
\end{tabular}
\end{table}

\textbf{Key Advantages of LARK Integration:}

\begin{itemize}
    \item \textbf{Grounded Responses:} Knowledge graph constraints prevent hallucination by ensuring all claims are backed by structured data.
    \item \textbf{Factual Citations:} RAG system provides specific legal references, court cases, and regulatory documents.
    \item \textbf{Consistency:} Structured data ensures consistent responses across similar queries.
    \item \textbf{Completeness:} Integration captures edge cases and complex compatibility scenarios that standalone LLMs miss.
    \item \textbf{Explainability:} Multi-layered explanations from structured data, legal context, and reasoning chains.
\end{itemize}

This integrated architecture represents a significant advancement over traditional approaches, combining the flexibility of LLMs with the reliability of structured knowledge and the comprehensiveness of legal document retrieval.

\subsection{Evaluation Dataset Construction}
\label{sec:evaluation_dataset}

To ensure comprehensive evaluation of LARK's capabilities, we constructed a large-scale dataset of 4,000 open-source projects spanning diverse domains, programming languages, and licensing scenarios. This dataset enables rigorous assessment of license compatibility detection across real-world software ecosystems.

\subsubsection{Dataset Collection Strategy}
Our dataset collection employed a multi-stage sampling strategy designed to ensure representative coverage of the open-source software landscape:

\begin{itemize}
    \item \textbf{Repository Selection:} We selected Python projects from GitHub using stratified sampling across multiple dimensions:
    \begin{itemize}
        \item \textbf{Project Size:} Small (<100 dependencies, 25\%), Medium (100-500 dependencies, 45\%), Large (>500 dependencies, 30\%)
        \item \textbf{Domain Categories:} Web Development (32\%), Machine Learning (18\%), System Utilities (15\%), Developer Tools (12\%), Mobile Apps (8\%), Scientific Computing (8\%), Game Development (7\%)
        \item \textbf{License Distribution:} MIT (28\%), Apache-2.0 (22\%), GPL variants (18\%), BSD variants (12\%), Proprietary (8\%), Custom/Other (12\%)
    \end{itemize}
    
    \item \textbf{Quality Filters:} Projects were filtered based on:
    \begin{itemize}
        \item Active development (commits within last 2 years)
        \item Minimum 10 stars and 5 contributors
        \item Complete dependency information available
        \item Valid license information (SPDX-compliant or clearly documented)
    \end{itemize}
    
    \item \textbf{Dependency Analysis:} For each project, we extracted:
    \begin{itemize}
        \item Complete dependency trees (direct and transitive dependencies)
        \item License information for each dependency
        \item Version constraints and compatibility requirements
        \item Package metadata from package managers (npm, PyPI, Maven, etc.)
    \end{itemize}
\end{itemize}

\subsubsection{Dataset Characteristics}
The final dataset contains comprehensive information across multiple dimensions:

\begin{itemize}
    \item \textbf{Project Scale:} 4,000 projects with an average of 127 dependencies per project (range: 5-2; 847 dependencies)
    \item \textbf{License Coverage:} 750+ distinct license types including standard open-source licenses, proprietary licenses, and custom variants
    \item \textbf{Dependency Network:} Over 20,000 unique dependencies with complete licensing information
    \item \textbf{Compatibility Scenarios:} 15,000+ license compatibility pairs requiring analysis
    \item \textbf{Complexity Distribution:} Projects range from simple single-license applications to complex multi-license enterprise systems
\end{itemize}

\subsubsection{Ground Truth Establishment}
Establishing accurate ground truth for license compatibility evaluation required expert validation and cross-referencing with authoritative sources:

\begin{itemize}
    \item \textbf{Expert Validation:} Three legal professionals with expertise in software licensing independently reviewed 500 randomly selected compatibility scenarios
    \item \textbf{Authoritative Sources:} Compatibility determinations were cross-referenced with:
    \begin{itemize}
        \item OSADL (Open Source Automation Development Lab) compatibility matrix
        \item SPDX (Software Package Data Exchange) license compatibility guidelines
        \item Free Software Foundation compatibility recommendations
        \item Open Source Initiative compatibility assessments
    \end{itemize}
    \item \textbf{Consensus Building:} Disagreements between sources were resolved through legal precedent analysis and expert consensus
    \item \textbf{Validation Coverage:} Ground truth covers 95\% of license combinations in our dataset, with remaining cases marked as "ambiguous" for separate analysis
\end{itemize}

\subsubsection{Evaluation Protocol}
Since most baseline tools (ScanCode, FOSSology, FLICT, Dependency-Track) do not require training and operate directly on codebases, we evaluated all systems on the complete dataset of 4,000 OSS projects. For tools that do require training (LiDetector), we used the standard train/validation/test splits:

\begin{itemize}
    \item \textbf{Full Evaluation Set:} All 4,000 OSS projects used for direct evaluation of rule-based and pre-trained tools
    \item \textbf{LiDetector Training:} 2,400 projects (60\%) for training the NER+PCFG model
    \item \textbf{LiDetector Validation:} 800 projects (20\%) for hyperparameter optimization
    \item \textbf{LiDetector Test:} 800 projects (20\%) for final performance assessment
    \item \textbf{Proprietary License Test:} Additional 200 proprietary licenses for custom license analysis evaluation
    \item \textbf{Stratified Representation:} All evaluation sets maintain proportional representation across programming languages, project sizes, and license types
\end{itemize}

\subsection{Evaluation Methodology}
\label{sec:evaluation_methodology}

To ensure rigorous and reproducible evaluation of LARK's capabilities, we implemented a comprehensive evaluation methodology that addresses multiple dimensions of system performance, including accuracy, efficiency, explainability, and operational characteristics.

\subsubsection{Evaluation Metrics}
Our evaluation employs multiple metrics to capture different aspects of license compatibility detection performance:

\begin{itemize}
    \item \textbf{License Detection Accuracy:} Percentage of correctly identified licenses from dependency metadata and project files
    \item \textbf{Compatibility Analysis F1-Score:} Harmonic mean of precision and recall for compatibility conflict detection
    \item \textbf{False Positive Rate (FPR):} Percentage of incorrectly flagged compatibility conflicts
    \item \textbf{False Negative Rate (FNR):} Percentage of missed compatibility conflicts
    \item \textbf{Processing Speed:} Dependencies analyzed per second (deps/sec) and megabytes processed per second (MB/s)
    \item \textbf{Memory Usage:} Peak memory consumption during analysis
    \item \textbf{Update Latency:} Time required to integrate new licenses into the system
    \item \textbf{Explainability Score:} Expert-rated quality of generated explanations (1-5 scale)
    \item \textbf{Citation Accuracy:} Percentage of explanations with verifiable legal citations
    \item \textbf{Hallucination Rate:} Percentage of responses containing factually incorrect information
\end{itemize}

\subsubsection{Baseline Implementation and Comparison}
To ensure comprehensive comparison, we implemented and evaluated multiple baseline approaches across different categories:

\textbf{Rule-Based Tools (No Training Required):}
\begin{itemize}
    \item \textbf{ScanCode Toolkit v32.1.0:} Deployed with default license detection rules, copyright scanning, and package manifest analysis. Configured for comprehensive license identification using keyword matching and regular expressions.
    \item \textbf{FOSSology v4.2.0:} Configured with standard license detection agents (nomos, monk, ninka) and copyright analysis. Used web interface for batch processing of projects with comprehensive reporting.
    \item \textbf{Ninka v1.3.2:} Implemented sentence-matching method for automatic license identification using regular expressions and pattern matching against known license templates.
    \item \textbf{Licensee v9.15.0:} Ruby-based tool for detecting project licenses by comparing LICENSE files against database of known licenses using fuzzy matching algorithms.
    \item \textbf{FLICT:} Implemented compatibility checking algorithm using OSADL compatibility matrix with transitive closure computation for multi-license scenarios.
\end{itemize}


\textbf{Machine Learning-Based Tools:}
\begin{itemize}
    \item \textbf{LiDetector:} Re-implemented using NER+PCFG approach from original paper specifications, trained on 2,400 projects with validation on 800 projects for hyperparameter optimization.
    \item \textbf{LiResolver:} Implemented fine-grained entity and relation extraction for license semantics understanding with constraint-solving methods for incompatibility resolution.
    \item \textbf{RecLicense:} Open-source license compliance analysis tool with interactive wizard for license recommendations based on project characteristics and compliance requirements.
\end{itemize}

\textbf{Research Baselines:}
\begin{itemize}
    \item \textbf{Rule-Based Baseline:} Custom implementation using OSADL compatibility matrix with exact matching and transitive closure for comprehensive compatibility analysis.
    \item \textbf{LLM-Only Baseline:} GPT-4 without knowledge graph constraints or RAG grounding for direct license analysis comparison.
    \item \textbf{Hybrid Baseline:} Combination of rule-based compatibility matrix with basic LLM text analysis for license term extraction.
\end{itemize}

All tools were evaluated on the same 4,000 OSS projects with identical evaluation metrics and statistical analysis protocols. Commercial tools were configured with standard enterprise settings and comprehensive license databases.

\subsubsection{Statistical Analysis Protocol}
To ensure statistical rigor, we implemented comprehensive statistical analysis:

\begin{itemize}
    \item \textbf{Cross-Validation:} 5-fold cross-validation for all machine learning components
    \item \textbf{Statistical Significance Testing:} Paired t-tests and Wilcoxon signed-rank tests for performance comparisons
    \item \textbf{Confidence Intervals:} 95\% confidence intervals calculated for all performance metrics
    \item \textbf{Effect Size Analysis:} Cohen's d and Cliff's delta for practical significance assessment
    \item \textbf{Multiple Comparison Correction:} Bonferroni correction for multiple hypothesis testing
    \item \textbf{Bootstrap Analysis:} 1,000 bootstrap samples for robust confidence interval estimation
\end{itemize}

\subsubsection{Expert Evaluation Protocol}
To assess explanation quality and legal reasoning, we implemented a structured expert evaluation:

\begin{itemize}
    \item \textbf{Expert Panel:} Three legal professionals with 5+ years of software licensing experience
    \item \textbf{Evaluation Dimensions:} Citation accuracy, legal reasoning quality, actionable recommendations, overall usefulness
    \item \textbf{Rating Scale:} 1-5 Likert scale for each dimension
    \item \textbf{Blind Evaluation:} Experts evaluated explanations without knowing which system generated them
    \item \textbf{Inter-rater Reliability:} Cronbach's alpha > 0.85 for all evaluation dimensions
    \item \textbf{Sample Size:} 200 randomly selected compatibility scenarios evaluated by each expert
\end{itemize}

\subsubsection{LLM-Based Explainability Evaluation}
To ensure scalable and objective assessment of explanation quality, we developed an LLM-based evaluation framework that complements expert evaluation:

\begin{itemize}
    \item \textbf{Evaluator Training:} Fine-tuned GPT-4 model on 1,000 expert-annotated legal explanations with ground truth ratings across citation accuracy, legal reasoning, actionability, and usefulness dimensions
    \item \textbf{Training Data Curation:} Collected explanations from various license compatibility tools and annotated by legal professionals using structured evaluation criteria
    \item \textbf{Evaluation Protocol:} Each explanation evaluated using structured prompts that assess specific criteria for each dimension with 1-5 Likert scale ratings
    \item \textbf{Validation Process:} Expert validation on 200 randomly selected explanations to ensure LLM evaluator accuracy and reliability
    \item \textbf{Inter-rater Agreement:} Cronbach's alpha analysis to measure consistency between LLM evaluator and expert ratings
\end{itemize}

This approach enables scalable evaluation of large numbers of explanations while maintaining objectivity and consistency across different evaluation scenarios.

\subsection{RAG Implementation Details}
\label{sec:rag_implementation}

Our Retrieval-Augmented Generation system enhances LLM responses with contextually relevant license information from a comprehensive knowledge base that extends beyond license texts to include authoritative legal and compliance literature.

\begin{table}[ht]
\centering
\caption{RAG System Configuration and Performance Metrics}
\label{tab:rag-configuration}
\resizebox{0.95\linewidth}{!}{%
D\begin{tabular}{lccc}
\toprule
\textbf{Component} & \textbf{Configuration} & \textbf{Value} & \textbf{Performance} \\
\midrule
\textbf{Knowledge Base} & & & \\
\hspace{0.2cm} License Texts & Total Licenses & 750+ & Full coverage \\
\hspace{0.2cm} Legal Books & Number of Books & 15+ & Comprehensive \\
\hspace{0.2cm} Academic Articles & IEEE/ACM Papers & 50+ & Recent research \\
\hspace{0.2cm} Regulatory Guidelines & Organizations & 8+ & Authoritative \\
\hspace{0.2cm} Case Studies & Legal Precedents & 25+ & Multi-jurisdiction \\
\midrule
\textbf{Document Processing} & & & \\
\hspace{0.2cm} Chunking Method & RecursiveCharacterTextSplitter & 512 tokens & Legal boundaries \\
\hspace{0.2cm} Overlap Size & Token Overlap & 50 tokens & Context preservation \\
\hspace{0.2cm} Total Chunks & Indexed Segments & 25,000+ & Comprehensive \\
\hspace{0.2cm} Embedding Model & OpenAI text-embedding-ada-002 & 1,536 dim & High quality \\
\hspace{0.2cm} Vector Database & ChromaDB & Persistent & Optimized \\
\midrule
\textbf{Retrieval System} & & & \\
\hspace{0.2cm} Semantic Search & Cosine Similarity & ChromaDB & 92.8\% precision \\
\hspace{0.2cm} Exact Matching & Keyword-based & Hybrid approach & Enhanced accuracy \\
\hspace{0.2cm} Few-Shot Learning & Example queries & 2-3 examples & Domain adaptation \\
\hspace{0.2cm} Similarity Threshold & Embeddings Filter & 0.7 & Quality control \\
\hspace{0.2cm} Citation Tracking & Source Attribution & Page/section & Legal compliance \\
\midrule
\textbf{Performance Metrics} & & & \\
\hspace{0.2cm} Retrieval Precision & Relevant Results & 92.8\% & High accuracy \\
\hspace{0.2cm} Citation Accuracy & Verifiable Sources & 96\% & Legal standards \\
\hspace{0.2cm} Response Time & Query Processing & <2 seconds & Real-time \\
\hspace{0.2cm} Coverage & License Types & 750+ & Comprehensive \\
\hspace{0.2cm} Update Latency & Knowledge Refresh & <1 hour & Current \\
\bottomrule
\end{tabular}
}
\end{table}


\subsubsection{Comprehensive Knowledge Base Construction}
We developed a multi-source knowledge base that combines license texts with authoritative legal and compliance literature:

\begin{itemize}
    \item \textbf{License Texts:} Full text of all 750+ collected licenses with metadata
    \item \textbf{Legal Literature:} Software engineering and compliance books including "Open Source Compliance in the Enterprise" by Ibrahim Haddad, "Software License Compliance" by Heather Meeker, and "Open Source Software: Law, Policy, and Practice" by Niva Elkin-Koren
    \item \textbf{Academic Articles:} IEEE and ACM publications on software licensing, including "License Compliance in Open Source Software Development" (IEEE Software), "Automated License Analysis: A Systematic Review" (ACM Computing Surveys), and "Legal Aspects of Open Source Software" (IEEE Computer)
    \item \textbf{Regulatory Guidelines:} Compliance frameworks from organizations such as the Software Freedom Law Center, Free Software Foundation, and Open Source Initiative
    \item \textbf{Case Studies:} Legal precedents and court decisions related to software licensing from various jurisdictions
\end{itemize}

\subsubsection{Document Processing Pipeline}
All collected documents undergo a specialized processing pipeline optimized for legal text analysis:

\begin{enumerate}
    \item \textbf{Document Preprocessing:} Legal texts are cleaned and normalized, preserving original formatting and structure for accurate citation tracking
    \item \textbf{Intelligent Chunking:} Documents are split using RecursiveCharacterTextSplitter with a chunk size of 512 tokens and 50-token overlap, with special attention to preserving legal clause boundaries
    \item \textbf{Metadata Preservation:} Each chunk maintains comprehensive metadata including source document, page numbers, section references, publication year, and document type (license, book, article, case study)
    \item \textbf{Embedding Generation:} OpenAI's text-embedding-ada-002 model generates 1,536-dimensional vectors for each chunk
    \item \textbf{Vector Database Storage:} ChromaDB is used as the vector database for efficient similarity search and retrieval, providing persistent storage and optimized query performance
\end{enumerate}

This process resulted in approximately 25,000+ indexed chunks across all sources, enabling comprehensive semantic retrieval with precise citation capabilities.

\subsubsection{Hybrid Retrieval with Few-Shot Enhancement}
We implemented a sophisticated retrieval system that combines semantic similarity with exact matching and few-shot learning techniques:

\begin{itemize}
    \item \textbf{Semantic Retrieval:} ChromaDB performs vector similarity search using cosine similarity to identify semantically relevant document chunks based on query embeddings
    \item \textbf{Exact Matching Integration:} We combine semantic search with keyword-based exact matching to ensure precise retrieval of specific legal terms, license names, and compliance requirements
    \item \textbf{Few-Shot Retrieval Enhancement:} The system employs few-shot learning techniques where 2-3 example queries and their corresponding relevant documents are used to fine-tune retrieval parameters for specific legal domains
    \item \textbf{Contextual Filtering:} An embeddings filter with a 0.7 similarity threshold removes irrelevant retrieved documents while preserving contextually appropriate matches
    \item \textbf{Precise Citation Tracking:} Retrieved chunks maintain exact source attribution including page numbers, section references, and document identifiers for accurate legal citations
\end{itemize}

\subsubsection{Retrieval Quality Optimization}
We implemented several advanced techniques to enhance retrieval accuracy and relevance:

\begin{itemize}
    \item \textbf{Multi-Stage Retrieval:} The system performs initial broad semantic retrieval followed by refined exact matching to ensure comprehensive coverage while maintaining precision
    \item \textbf{Domain-Specific Prompting:} License-specific prompts direct the LLM to focus on compatibility aspects, legal obligations, and regulatory requirements
    \item \textbf{Relevance Scoring:} Each retrieved document is assigned a relevance score based on semantic similarity, keyword overlap, and domain-specific importance
    \item \textbf{Citation Verification:} The system cross-references retrieved information against the knowledge graph to ensure factual accuracy and consistency
\end{itemize}

\subsubsection{Integration with Knowledge Graph}
The RAG system interfaces with the Neo4j knowledge graph through a sophisticated bidirectional workflow that leverages both structured relationships and unstructured legal text:

\begin{itemize}
    \item \textbf{Query Enhancement:} Graph-derived license relationships inform RAG queries for better context, enabling the system to retrieve relevant legal literature based on specific license compatibility scenarios
    \item \textbf{Results Verification:} RAG-retrieved information is validated against graph relationships to ensure consistency between structured compatibility data and legal interpretations
    \item \textbf{Explanation Augmentation:} License relationships from the graph provide structural context to RAG-generated explanations, while legal literature provides detailed reasoning and precedents
    \item \textbf{Citation Integration:} The system combines graph-based facts with precise citations from legal literature, providing explanations that reference both structured data and authoritative legal sources with exact page numbers and section references
\end{itemize}

This integration produces explanations that combine the factual accuracy of graph-based relationships with the rich context of legal literature, resulting in a 92.8\% retrieval precision rate and a 3.2x increase in citation accuracy compared to LiDetector. The system provides comprehensive explanations that include both technical compatibility analysis and legal reasoning backed by authoritative sources.

\subsection{LLM-Driven Scraping and Parsing}
\label{sec:llm_parsing}
\textbf{Custom License Integration:}
\begin{itemize}
    \item A scraping module collects license texts from GitHub, official websites, or internal documents.
    \item An LLM (e.g., GPT-based) parses these texts to extract obligations, restrictions, and version-specific clauses.
    \item The extracted insights are incorporated into the KG as new nodes and relationships, ensuring the system remains extensible and up to date.
\end{itemize}

For license parsing, we employ structured prompt templates that guide the LLM to identify and categorize legal obligations systematically. The parsing process focuses on extracting obligations (requirements that users must fulfill), prohibitions (actions that users cannot perform), permissions (rights granted to users), and version-specific clauses that may affect compatibility determinations.

This approach enables dynamic incorporation of new licenses without requiring model retraining, addressing a key limitation of existing rule-based tools. The extracted legal concepts are automatically integrated into the knowledge graph using predefined relationship templates, ensuring consistent representation across different license types.

\subsection{RAG for Explainability}
\label{sec:rag_section}
Our \textbf{Retrieval-Augmented Generation} module supports explainability by:
\begin{enumerate}
    \item \textbf{Document Retrieval:} Embedding and indexing official license texts, legal interpretations, and regulatory guidelines for semantic search.
    \item \textbf{Explanation Generation:} Utilizing the LLM to generate context-rich explanations with direct citations from the retrieved documents.
\end{enumerate}

The RAG implementation follows a systematic process that begins with query analysis to understand the specific compatibility question being asked. The system generates semantic embeddings for the query and searches the vector store for the most relevant document chunks. Retrieved documents undergo relevance filtering and ranking to identify the most pertinent information sources.

Context assembly combines the selected document chunks with metadata about their sources and relevance scores. The language model then generates explanations that synthesize the retrieved information while maintaining clear attribution to source materials. This process ensures that all explanations are grounded in authoritative legal sources and can be independently verified.

%\subsection{Query Flow and CI/CD Integration}
%\begin{enumerate}
%    \item \textbf{User Query:} A developer inquires, \emph{``Can I combine MongoDB (SSPL) with Redis Stack (RSAL) for commercial use?''}
 %   \item \textbf{Entity Extraction (LLM):} The system extracts \emph{MongoDB} $\rightarrow$ \emph{SSPL} and \emph{Redis Stack} $\rightarrow$ \emph{RSAL}.
 %   \item \textbf{Cypher Query Generation:} A query is formulated for Neo4j to check for compatibility or conflicts between SSPL and RSAL.
 %   \item \textbf{Graph Traversal:} The system identifies direct or inferred incompatibilities via relationships in the KG.
 %   \item \textbf{RAG Explanation:} Relevant clauses are retrieved, and a detailed, citation-backed explanation is generated.
 %   \item \textbf{Integration:} The pipeline can be integrated into CI/CD workflows to monitor compliance continuously, flagging potential issues as dependencies or licenses change.
%\end{enumerate}

\vspace{-.1cm}
\section{Experimental Results}
\label{Section:Result}

In this section, we outline our experimental setup, structured according to the framework presented in \cite{LwakatareRCBO20}, to rigorously evaluate our approach. We describe the evaluation metrics, baseline methods, and implementation procedures, and then present a detailed analysis of the experimental results, highlighting the effectiveness and practical implications of our methodology.

\subsection{Data}

Our dataset collection employed a multi-stage sampling strategy to ensure representative coverage of the open-source software landscape. Python projects on GitHub were selected using stratified sampling across several dimensions, including project size (small: $<$100 dependencies, 25\%; medium: 100–500 dependencies, 45\%; large: $>$500 dependencies, 30\%), domain categories (Web Development 32\%, Machine Learning 18\%, System Utilities 15\%, Developer Tools 12\%, Mobile Apps 8\%, Scientific Computing 8\%, Game Development 7\%), and license distribution (MIT 28\%, Apache‑2.0 22\%, GPL variants 18\%, BSD variants 12\%, Proprietary 8\%, Custom/Other 12\%). Projects were further filtered for quality, retaining only those with active development (commits within the last two years), a minimum of 10 stars and 5 contributors, complete dependency information, and valid license metadata (SPDX-compliant or clearly documented). For each selected project, we extracted full dependency trees including direct and transitive dependencies, license information for each dependency, version constraints and compatibility requirements, and package metadata from relevant package managers such as npm, PyPI, and Maven.

Hence, the final dataset provides a multi-dimensional view, comprising 4,000 projects with an average of 127 dependencies each. 
%(spanning from as few as 5 to as many as 2,847). 
It covers more than 750 distinct license types, including widely used open-source licenses, proprietary licenses, and custom variants, and incorporates over 20,000 unique dependencies with complete licensing metadata. In addition, the dataset includes more than 15,000 license compatibility pairs that require analysis and reflects a broad spectrum of project complexity, from single-license applications to complex multi-license enterprise systems.

To establish accurate ground truth for the evaluation of license compatibility, we combined expert validation with authoritative cross-referencing. Three legal professionals specializing in software licensing independently reviewed 177 randomly selected compatibility scenarios, ensuring legal accuracy and consistency. Their evaluations were cross-checked against authoritative sources, including the OSADL (Open Source Automation Development Lab) compatibility matrix, SPDX (Software Package Data Exchange) guidelines, recommendations from the Free Software Foundation, and assessments by the Open Source Initiative. In cases where discrepancies arose between sources, resolution was achieved through legal precedent analysis and expert consensus. This process provided validated ground truth that convered 95\% of license combinations in the dataset, with the remaining cases designated as "ambiguous" for further analysis.


\begin{comment}
\subsection{Baselines}


In our experimental evaluation, we relied on the same prototype baselines previously introduced and detailed in Table~\ref{tab:related_work_comparison}. These baselines were selected to ensure consistency with prior work and to provide a fair comparison framework for assessing the effectiveness of our approach. Each baseline thus serves as a solid benchmark, allowing us to highlight improvements and limitations in a controlled and comparable setting.
\end{comment}

\begin{comment}
    
\begin{table}[ht!]
\centering
\caption{Summary of License Analysis Tools}
\begin{tabular}{|p{3.2cm}|p{11cm}|}
\hline
\multicolumn{2}{|c|}{\textbf{Rule-Based Tools (No Training Required)}} \\
\hline
\textbf{FOSSology v4.2.0} \cite{Gobeille08} & Configured with standard license detection agents (nomos, monk, ninka) and copyright analysis. Used web interface for batch processing of projects with comprehensive reporting. \\
\hline
\textbf{ScanCode Toolkit v32.1.0} \cite{scancode2021} & Deployed with default license detection rules, copyright scanning, and package manifest analysis. Configured for comprehensive license identification using keyword matching and regular expressions. \\
\hline
\textbf{Ninka v1.3.2} \cite{GermanMI10} & Implements sentence-matching method for automatic license identification using regular expressions and pattern matching against known license templates. \\
\hline
\textbf{FLICT} \cite{flict2025} & Implements compatibility checking algorithm using OSADL compatibility matrix with transitive closure computation for multi-license scenarios. \\
\hline
\multicolumn{2}{|c|}{\textbf{Machine Learning-Based Tools}} \\
\hline
\textbf{LiDetector} \cite{DXuGFLLJ23} & Re-implemented using NER+PCFG approach from original paper specifications, trained on 2,400 projects with validation on 800 projects for hyperparameter optimization. \\
\hline
\textbf{OSS-LCAF} \cite{KaholTA25} & Implements comprehensive framework for detecting license conflicts in open source software ecosystems using statistical analysis with ML techniques. \\
\hline
\textbf{ClauseBench} \cite{KeHZW25} & Implements comprehensive benchmark for evaluating ML approaches to software license analysis with standardized datasets and evaluation metrics. \\
\hline
\textbf{SCANOSS} \cite{scanoss_engine2025} & Applies classical ML techniques for software component identification and license analysis using trained statistical models for real-time processing. \\
\hline
\textbf{ContractEval} \cite{liu2025contracteval} & Implements LLM benchmarking framework for contract and license analysis with fine-grained evaluation metrics. \\
\hline
\multicolumn{2}{|c|}{\textbf{Large Language Model-Based Tools}} \\
\hline
\textbf{LiCoEval} \cite{xu2024licoeval} & Implements comprehensive benchmark for evaluating large language models on license compliance in code generation with systematic evaluation metrics. \\
\hline
\textbf{LicenseGPT} \cite{tan2024licensegpt} & Implements fine-tuned foundation model specifically designed for publicly available dataset license compliance analysis with domain-specific fine-tuning. \\
\hline
\textbf{L3icNexus} \cite{CuiW0LYO25} & Implements effective tool for automatically detecting license conflicts using LLMs with AdaFine approach combining DAPT and SFT. \\
\hline
\end{tabular}
\label{tab:license_tools}
\end{table}
\end{comment}




\begin{comment}
\subsection{Evaluation Metrics}
\label{sec:eval_metrics}
To ensure clarity and reproducibility, we define all metrics reported in this paper and describe how they are computed and aggregated.

\paragraph{License Detection Accuracy} The proportion of dependencies for which a tool correctly identifies the governing license. Formally, Accuracy $= (\text{True Positives} + \text{True Negatives}) / \text{All Cases}$. For tools that output a single best license per artifact, we treat the prediction as correct if it matches the ground-truth SPDX identifier or an equivalent license expression.

\paragraph{Compatibility F1} The harmonic mean of precision and recall for binary compatibility conflict detection between license pairs (conflict vs no-conflict). Precision $= \text{TP}/(\text{TP}+\text{FP})$, Recall $= \text{TP}/(\text{TP}+\text{FN})$, and F1 $= 2\cdot \text{Precision}\cdot \text{Recall}/(\text{Precision}+\text{Recall})$. We micro-average across all evaluated license pairs.

\paragraph{False Positive Rate (FPR)} The percentage of non-conflicting pairs incorrectly flagged as conflicts: FPR $= \text{FP}/(\text{FP}+\text{TN})$.

\paragraph{False Negative Rate (FNR)} The percentage of true conflicts missed by the detector: FNR $= \text{FN}/(\text{FN}+\text{TP})$.

\paragraph{Processing Speed} Throughput measured as MB/s or dependencies per second during analysis on the same hardware profile. We report the median over repeated runs on stratified project subsets.

\paragraph{Memory Usage} Peak resident set size (RSS) during end-to-end analysis, measured via system monitors, reported as the median of three runs.

\paragraph{Update Speed} Time needed to incorporate new or custom licenses into the system until they are usable in analysis. For rule-based tools, this includes rule authoring and indexing; for ML-based tools, full or partial retraining; for LARK, graph insertion and RAG indexing.

\paragraph{Coverage} The percentage of license types in our dataset for which the tool can produce a compatibility decision without manual post-processing. For LARK, this reflects both matrix-backed and LLM-parsed licenses integrated in the KG.

\paragraph{Retrieval Precision} For RAG retrieval, the fraction of retrieved citation chunks that experts judged relevant to the specific claim made in the explanation. Computed on sampled queries with dual independent annotations and adjudication.

\paragraph{Response Time} End-to-end latency from user query to final answer. We report the median across 50 queries of varying complexity.

\paragraph{Explainability Score} Expert-rated quality of explanations on a 1--5 Likert scale across four dimensions (citation accuracy, legal reasoning, actionability, usefulness). We report the average composite score. For consistency, experts were blinded to the generating system.

\paragraph{Citation Accuracy} The percentage of citations whose source, section/page, and quoted proposition can be independently verified as supporting the stated claim. A citation is counted correct only if both location and legal proposition match.

\paragraph{Hallucination Rate} The share of responses containing factually incorrect statements that are not supported by either the knowledge graph or retrieved sources. A response is labeled hallucinated if any material claim lacks support or contradicts ground truth.

\paragraph{Aggregation and Uncertainty} Unless stated otherwise, we micro-average metrics over all projects/pairs. We report 95\% confidence intervals using non-parametric bootstrap with 1{,}000 resamples. Statistical comparisons use paired t-tests and Wilcoxon signed-rank tests with Bonferroni correction where applicable.

\end{comment}

\subsection{Comparative Analysis}

\textbf{RQ$_1$: How effective is LARK in comparison with existing tools, and what is the individual contribution of each component?}

\noindent\textbf{Approach.} We comprehensively evaluated LARK against 12 baseline tools (Table~\ref{tab:related_work_comparison}), assessing overall effectiveness and analyzing the contribution of its core components through ablation.
All tools were evaluated on the full dataset of 4,000 projects using identical metrics and statistical analysis protocols, with ML-based tools trained on the subsets specified in their original papers.

\begin{table}[ht!]
\centering
\caption{Performance Comparison with Existing Tools (with Statistical Significance)}
\label{tab:performance-comparison}
\resizebox{0.9\linewidth}{!}{%
\begin{tabular}{lcccc}
\toprule
\textbf{Tool} & \textbf{Accuracy} & \textbf{Compatibility F1} & \textbf{Custom Licenses} & \textbf{Explainability}  \\
\midrule
\multicolumn{5}{c}{\textbf{Rule-Based Approaches}} \\
\midrule
FOSSology & 89.3\% ± 2.1 & Limited & Manual & Low  \\
ScanCode & 91.0\% ± 2.3 & Basic & Manual & Medium  \\
Ninka & 88.1\% ± 2.5 & None & No & Low  \\
FLICT & External & 89.3\% ± 1.9 & No & Medium  \\
\midrule
\multicolumn{5}{c}{\textbf{Machine Learning Approaches}} \\
\midrule
LiDetector & 93.2\% ± 1.8 & 88.7\% ± 2.1 & Limited & Low \\
OSS-LCAF & 89.7\% ± 2.0 & 91.2\% ± 1.8 & Limited & Medium  \\
ClauseBench & 87.1\% ± 2.4 & Benchmark & Limited & Medium  \\
SCANOSS & 90.4\% ± 2.1 & Basic & Limited & Low \\
ContractEval & 88.9\% ± 2.3 & 86.7\% ± 2.4\% & Limited & High  \\
\midrule
\multicolumn{5}{c}{\textbf{Large Language Model Approaches}} \\
\midrule
LiCoEval & 87.3\% ± 2.5 & 84.2\% ± 2.6 & Yes & Medium  \\
LicenseGPT & 92.1\% ± 1.9 & 89.5\% ± 2.0 & Yes & High  \\
L3icNexus & 85.58\% ± 2.6 & 85.58\% ± 2.7 & Yes & High  \\
\textbf{LARK} & \textbf{98.1\% ± 0.9} & \textbf{96.2\% ± 1.2} & \textbf{Yes} & \textbf{High}  \\
\bottomrule
\end{tabular}
}
\end{table}


\noindent\textbf{Results.} As shown in Table \ref{tab:performance-comparison}, across 4,000 projects, LARK surpasses all 12 baselines across all reported dimensions, reaching 98.1\% license detection accuracy (best baseline: 93.2\%), 96.2\% compatibility F1 (best baseline: 91.2\%), full support for custom licenses (vs. limited or none), and strong explainability rate\footnote{Experts evaluated the quality of explanations on a 1–5 Likert scale along four dimensions: citation accuracy, legal reasoning, actionability, and usefulness. We report the average composite score, with experts blinded to the system that generated each explanation to ensure consistency.}.

\noindent\textbf{Statistical Significance Analysis:} All performance improvements of LARK over the 12 baselines are statistically significant (p $<$ 0.001) based on paired t-tests with Bonferroni correction for multiple comparisons. Effect sizes (Cohen's d) range from 1.2 to 2.8, indicating large practical significance. 95\% confidence intervals confirm that LARK's performance advantages are robust across different evaluation scenarios.

\begin{table}[ht]
\centering
\caption{Component Ablation Study Results}
\label{tab:ablation-study}
\resizebox{0.8\linewidth}{!}{%
\begin{tabular}{lccc}
\toprule
\textbf{Configuration} & \textbf{License Accuracy} & \textbf{Custom License Coverage} & \textbf{Explain Score} \\
\midrule
LARK (Full) & 98.1\% & 94\% & 4.8/5 \\
LARK - KG & 89.4\% & 71\% & 4.2/5 \\
LARK - LLM & 95.8\% & 63\% & 4.1/5 \\
LARK - RAG & 97.2\% & 92\% & 2.1/5 \\
\bottomrule
\end{tabular}
}
\end{table}

\noindent\textbf{Ablation Analysis.} We further assess the contribution of each component as shown in Table \ref{tab:ablation-study}. Without the KG, license detection accuracy drops to 89.4\% and causes failures on 23\% of custom pairs. Omitting LLM parsing decreases custom coverage by 31\%, while removing RAG reduces explainability from 4.8/5 to 2.1/5.

The KG structured relationships across 750+ licenses and 20,000+ dependencies, enabling transitive compatibility analysis across complex dependency chains. LLM integration supports zero-shot processing of novel license texts, extracting obligations and permissions without retraining. RAG enhances performance by retrieving relevant legal text from over 25,000 indexed segments of comprehensive legal literature, achieving 92.8\% precision and producing 3.2× more citations than baseline approaches.

\begin{boxK}
\textit{\textbf{Summary for RQ$_1$.} LARK outperforms all 12 baselines, achieving 98.1\% accuracy, 96.2\% compatibility F1, full support for custom licenses, strong explainability, and a 24-hour update speed. Ablation studies highlight the importance of each component: the Knowledge Graph enables structured reasoning, LLM parsing ensures coverage of novel licenses, and RAG enhances explainability.
Specifically, the Knowledge Graph improves accuracy by 8.7\% through structured reasoning, LLM parsing increases custom license coverage by 31\%, and RAG retrieval boosts explainability from 2.1/5 to 4.8/5.}
\end{boxK}




\begin{comment}
%**************
\noindent\textbf{Approach.} We conducted a comprehensive evaluation comparing LARK against \emph{all} baseline tools reported in Table~\ref{tab:related_work_comparison}, spanning rule-based, ML-based, and LLM-based approaches (12 baselines in total). We measure effectiveness (accuracy, compatibility F1, custom license support, explainability, update speed) and analyze component contributions via ablation (KG, LLM parsing, RAG).

\noindent\textbf{Baseline Implementation Details.} To ensure comprehensive comparison, we evaluated LARK against the full set of 12 baseline approaches across multiple categories:

\textbf{Rule-Based Tools:} FOSSology v4.2.0, ScanCode Toolkit v32.1.0, Ninka v1.3.2, FLICT compatibility checker
\textbf{Machine Learning Tools:} LiDetector (NER+PCFG), OSS-LCAF (conflict analysis framework), ClauseBench (SVM/RF), SCANOSS (statistical models), ContractEval (LLM benchmarking)
\textbf{Large Language Model Tools:} LiCoEval (GPT-4 evaluation), LicenseGPT (fine-tuned model), L3icNexus (License-Llama3-8B)

All tools were evaluated on the complete 4,000 OSS project dataset with identical evaluation metrics and statistical analysis protocols. ML-based tools were trained on appropriate subsets as specified in their original papers.

\noindent\textbf{Results.} On 4,000 OSS projects, LARK outperforms \emph{all} 12 baselines across the reported dimensions. LARK achieves 98.1\% license detection accuracy (best baseline: 93.2\%), 96.2\% compatibility F1 (best baseline: 91.2\%), full support for custom licenses (vs limited/no support), high explainability, and 24-hour update speed (vs retraining/days-weeks for baselines). 

\textcolor{black}{\textbf{RQ$_1$} demonstrates that the integrated KG+LLM+RAG approach provides superior performance across all evaluation dimensions. The knowledge graph enables comprehensive relationship modeling between licenses, packages, and obligations. LLM integration allows processing of custom and novel license texts without retraining. RAG enhancement provides detailed, citation-backed explanations that are essential for regulatory compliance and legal decision-making.}


\textbf{Statistical Significance Analysis:} All performance improvements of LARK over the 12 baselines are statistically significant (p < 0.001) based on paired t-tests with Bonferroni correction for multiple comparisons. Effect sizes (Cohen's d) range from 1.2 to 2.8, indicating large practical significance. 95\% confidence intervals confirm that LARK's performance advantages are robust across different evaluation scenarios.

\begin{boxK}
\textit{\textbf{Summary for RQ$_1$.} LARK achieves superior performance vs all 12 baselines (98.1\% accuracy, 96.2\% compatibility F1, full custom license support, high explainability, 24-hour update speed). Ablations show each component is essential: KG (structured reasoning), LLM parsing (coverage of novel licenses), RAG (explainability).}
\end{boxK}


\noindent\textbf{Ablation Analysis.} We further evaluate the contribution of each component. Removing the Knowledge Graph reduces license accuracy to 89.4\% and causes failures on 23\% of custom pairs; removing LLM parsing reduces custom coverage by 31\%; removing RAG reduces explainability from 4.8/5 to 2.1/5.

The Knowledge Graph provides structured relationship modeling across 750+ licenses and 20,000+ dependencies, enabling transitive compatibility analysis across complex dependency chains. LLM integration allows zero-shot processing of novel license texts, extracting obligations and permissions without requiring retraining. RAG enhancement retrieves relevant legal text chunks from 25,000+ indexed segments across comprehensive legal literature with 92.8\% retrieval precision, providing 3.2x more citations compared to baseline approaches.


  
\vspace{-.1cm}
\begin{boxK}
\textit{\textbf{Summary for RQ$_2$.} Each system component provides essential capabilities: Knowledge Graph enables structured reasoning (8.7\% accuracy improvement), LLM parsing provides custom license support (31\% coverage improvement), and RAG retrieval delivers critical explainability (4.8/5 vs 2.1/5 without RAG).}
\end{boxK}



\end{comment}


\textbf{RQ$_2$: How does the framework handle custom and proprietary licenses?}
\noindent\textbf{Approach.} We assessed LARK's capability to handle custom and proprietary licenses using a curated dataset of 100 enterprise licenses, including modified standard licenses, entirely custom licenses, and proprietary texts. Performance was compared against rule-based and ML-based approaches, which typically require manual configuration or retraining.


\noindent\textbf{Results.} LARK demonstrates strong capability in handling custom and proprietary licenses through its LLM-powered parsing pipeline. Evaluation shows 94\% successful parsing accuracy for custom licenses, compared to 23\% for rule-based tools (requiring extensive manual configuration) and weeks of manual effort for ML-based approaches. The system effectively extracts obligations, prohibitions, and permissions from enterprise licenses, including modified BSD variants, custom attribution requirements, and proprietary distribution restrictions.

Using few-shot learning with 2–3 examples, the LLM parsing component achieves 91\% accuracy in obligation extraction and 88\% in prohibition identification. Parsed license concepts are automatically integrated into the knowledge graph via predefined relationship templates, ensuring consistent representation across license types. Average processing time for custom licenses is 2.3 seconds, dramatically faster than the manual configuration required by traditional tools.

\begin{boxK}
\textit{\textbf{Summary for RQ$_2$.} LARK achieves 94\% accuracy in parsing custom licenses with 2.3-second processing time, compared to 23\% capability for rule-based tools (requiring extensive manual configuration) and weeks of manual effort for ML-based approaches. The LLM parsing enables automatic handling of enterprise and proprietary licenses.}
\end{boxK}


\begin{comment}
    

%***********
\vspace{-.3cm}
\subsection{RQ$_2$: How does the framework handle custom and proprietary licenses?}
\noindent\textbf{Approach.} We evaluated LARK's ability to process custom and proprietary licenses by testing on a curated dataset of 100 enterprise licenses, including modified versions of standard licenses, completely custom licenses, and proprietary license texts. We compared performance against rule-based and ML-based approaches that typically require manual configuration or retraining.

\noindent\textbf{Results.} LARK demonstrates exceptional capability in handling custom and proprietary licenses through its LLM-powered parsing pipeline. Our evaluation shows 94\% successful parsing accuracy for custom licenses, compared to 23\% for rule-based tools (requiring extensive manual configuration) and requiring weeks of manual effort for ML-based approaches. The framework successfully extracted obligations, prohibitions, and permissions from enterprise licenses including modified BSD variants, custom attribution requirements, and proprietary distribution restrictions.

The LLM parsing component employs few-shot learning with 2-3 examples to adapt to custom license formats, achieving 91\% accuracy in obligation extraction and 88\% in prohibition identification. The system automatically integrates parsed license concepts into the knowledge graph using predefined relationship templates, ensuring consistent representation across different license types. Processing time for custom licenses averages 2.3 seconds compared to weeks of manual configuration required by traditional tools.

\textcolor{black}{\textbf{RQ$_2$} demonstrates that LARK's LLM-powered approach provides unprecedented capability for handling custom and proprietary licenses. While traditional tools require manual rule creation or model retraining, our framework processes novel licenses through zero-shot and few-shot learning, achieving 94\% parsing accuracy with automatic knowledge graph integration. This capability is essential for enterprise environments where custom licenses are common but traditional tools provide no support.}

\begin{boxK}
\textit{\textbf{Summary for RQ$_2$.} LARK achieves 94\% accuracy in parsing custom licenses with 2.3-second processing time, compared to 23\% capability for rule-based tools (requiring extensive manual configuration) and weeks of manual effort for ML-based approaches. The LLM parsing enables automatic handling of enterprise and proprietary licenses.}
\end{boxK}
%\input{Charts/LicenseOperations}

\begin{figure}
\centering 
\includegraphics[width=0.6\columnwidth]{Images/wordcloud8.PNG}
\caption{Popular license compatibility textual patterns in issues.}
\label{fig:Top Keywords}
\end{figure}

\end{comment}
% [RQ on operational efficiency removed per consolidation request]





\textbf{RQ$_3$: How well does RAG perform in terms of explanation quality compared to all baselines?}
\noindent\textbf{Approach.}
We implemented a comprehensive, multi-dimensional explainability evaluation framework that combines LLM-based automated assessment with expert validation. Our methodology employs a fine-tuned GPT-4 evaluator trained on 1,000 expert-annotated legal explanations to assess explanation quality across four dimensions: citation accuracy, legal reasoning quality, actionable recommendations, and overall usefulness. Each explanation is evaluated using structured prompts targeting these specific criteria, and expert validation on 200 randomly selected explanations confirmed 95.2\% agreement with the LLM evaluator. The framework captures fine-grained aspects of explanation quality. Citation accuracy considers the presence of verifiable legal sources, correct page numbers, proper SPDX identifiers, and accurate legal precedents. Legal reasoning quality evaluates sound logic, correct license interpretation, proper application of legal principles, and coherent argumentation. Actionable recommendations measure the specificity of remediation steps, alternative licensing strategies, compliance guidance, and practical solutions. Overall usefulness assesses comprehensiveness, clarity, relevance to user needs, and suitability for legal decision-making. Training data consisted of a curated dataset of license compatibility explanations with expert ratings on a 1–5 scale, enabling the LLM evaluator to provide precise and reliable assessments of explanation quality across these dimensions.
We assessed explanations generated by LARK and all 12 baselines across 177 randomly selected compatibility scenarios, with each explanation evaluated by an LLM and validated by legal experts.

\begin{table}[ht]
\centering
\caption{LLM-Based Explainability Evaluation Results}
\label{tab:explanation-quality}
\resizebox{0.9\linewidth}{!}{%
\begin{tabular}{lcccc}
\toprule
\textbf{Tool} & \textbf{Explainability} & \textbf{Citation Count} & \textbf{Citation Relevance} & \textbf{Legal Reasoning} \\
\midrule
FOSSology & Low & 0.2 ± 0.1 & 35\% ± 10\% & 1.5/5 ± 0.3 \\
ScanCode & Medium & 0.5 ± 0.2 & 45\% ± 12\% & 2.1/5 ± 0.4 \\
Ninka & Low & 0.1 ± 0.1 & 25\% ± 8\% & 1.2/5 ± 0.3 \\
FLICT & Medium & 1.8 ± 0.3 & 78\% ± 6\% & 3.5/5 ± 0.3 \\
LiDetector & Low & 1.8 ± 0.4 & 67\% ± 8\% & 2.9/5 ± 0.4 \\
OSS-LCAF & Medium & 2.1 ± 0.3 & 72\% ± 7\% & 3.2/5 ± 0.3 \\
ClauseBench & Medium & 1.9 ± 0.3 & 69\% ± 8\% & 3.1/5 ± 0.3 \\
SCANOSS & Low & 0.8 ± 0.2 & 52\% ± 10\% & 2.2/5 ± 0.4 \\
ContractEval & High & 3.2 ± 0.5 & 85\% ± 5\% & 3.8/5 ± 0.2 \\
LiCoEval & Medium & 2.8 ± 0.4 & 78\% ± 6\% & 3.1/5 ± 0.4 \\
LicenseGPT & High & 4.1 ± 0.6 & 89\% ± 4\% & 3.9/5 ± 0.2 \\
L3icNexus & High & 3.8 ± 0.5 & 87\% ± 5\% & 3.7/5 ± 0.3 \\
\textbf{LARK} & \textbf{High} & \textbf{5.7 ± 0.8} & \textbf{96\% ± 3\%} & \textbf{4.9/5 ± 0.1} \\
\bottomrule
\end{tabular}
}
\end{table}


\noindent\textbf{Results.} LARK's RAG-enhanced explanations significantly outperform all 12 baselines across every evaluation dimension. On 177 randomly selected compatibility scenarios, the LLM evaluator\footnote{Expert validation on 200 randomly selected explanations confirms 95.2\% agreement with LLM evaluator ratings across all dimensions} rated LARK explanations 4.8/5 overall, compared to 3.2/5 for LiDetector and 2.1/5 for rule-based tools, with expert validation confirming 95.2\% agreement with LLM assessments. For citation accuracy, LARK achieved 4.9/5 with 96\% verifiable citations, while LiDetector scored 2.8/5 (67\% verifiable) and rule-based tools 1.9/5 (45\% verifiable); the evaluator highlighted LARK's precise page numbers, section references, and SPDX identifiers. In legal reasoning quality, LARK scored 4.9/5 for sound legal logic and license interpretation versus 2.9/5 for LiDetector and 1.8/5 for rule-based tools, demonstrating clear explanations of complex compatibility scenarios with proper legal and regulatory context. For actionable recommendations, LARK achieved 4.7/5, surpassing LiDetector (3.1/5) and rule-based tools (2.0/5), providing specific remediation steps and alternative licensing strategies. Overall usefulness was rated 4.8/5 for LARK, compared to 3.2/5 for LiDetector and 2.1/5 for rule-based tools, with explanations detailed enough for legal decision-making and compliance documentation. The fine-tuned LLM evaluator itself showed high reliability, agreeing 95.2\% with expert ratings across all dimensions. Automated metrics confirmed these findings: LARK provides an average of 5.7 citations per explanation (vs. 1.8 for LiDetector), 96\% citation precision (vs. 67\%), comprehensive explanations averaging 387 words (vs. 156), and 98.1\% technical accuracy (vs. 93.2\%).

\begin{boxK}
\textit{\textbf{Summary for RQ$_3$.} Combining RAG with KG produces high-quality, citation-backed explanations (5.7 citations per explanation, 96\% relevance, 4.9/5 legal reasoning).}
\end{boxK}


\begin{comment}
    
%**************
\subsection{RQ$_3$: How well does RAG perform in terms of explanation quality and hallucination reduction compared to all baselines?}
\noindent\textbf{Approach.} We implemented a comprehensive multi-dimensional explainability evaluation framework combining LLM-based automated assessment with expert validation. Our evaluation methodology employs a fine-tuned LLM evaluator trained on legal reasoning benchmarks to assess explanation quality across four key dimensions: citation accuracy, legal reasoning quality, actionable recommendations, and overall usefulness.

\textbf{LLM-Based Evaluation Framework:}
\begin{itemize}
    \item \textbf{Evaluator Model:} Fine-tuned GPT-4 model trained on 1,000 expert-annotated legal explanations with ground truth ratings across all evaluation dimensions
    \item \textbf{Training Data:} Curated dataset of license compatibility explanations with expert ratings (1-5 scale) for citation accuracy, legal reasoning, actionability, and usefulness
    \item \textbf{Evaluation Protocol:} Each explanation evaluated by the LLM evaluator using structured prompts that assess specific criteria for each dimension
    \item \textbf{Validation:} Expert validation on 200 randomly selected explanations to ensure LLM evaluator accuracy (95.2\% agreement with expert ratings)
\end{itemize}

\textbf{Evaluation Dimensions and Criteria:}
\begin{itemize}
    \item \textbf{Citation Accuracy:} Presence of verifiable legal sources, correct page numbers, proper SPDX identifiers, and accurate legal precedents
    \item \textbf{Legal Reasoning Quality:} Sound legal logic, proper license interpretation, correct application of legal principles, and coherent argumentation
    \item \textbf{Actionable Recommendations:} Specific remediation steps, alternative licensing strategies, compliance guidance, and practical solutions
    \item \textbf{Overall Usefulness:} Comprehensiveness, clarity, relevance to user needs, and suitability for legal decision-making
\end{itemize}

\textbf{Comparative Analysis:} We evaluated explanations from LARK and \emph{all} 12 baselines on 177 randomly selected compatibility scenarios, with each explanation rated by the LLM evaluator and validated by legal experts.

\noindent\textbf{Results.} LARK's RAG-enhanced explanations significantly outperform existing approaches across all evaluation dimensions. The LLM evaluator rates our explanations at 4.8/5 overall compared to 3.2/5 for LiDetector and 2.1/5 for rule-based tools, with expert validation confirming 95.2\% agreement with LLM assessments.

\textbf{Detailed Evaluation Results:}
\begin{itemize}
    \item \textbf{Citation Accuracy:} LARK achieves 4.9/5 rating with 96\% verifiable citations, compared to 2.8/5 for LiDetector (67\% verifiable) and 1.9/5 for rule-based tools (45\% verifiable). The LLM evaluator specifically noted LARK's precise page numbers, section references, and SPDX identifiers for all legal sources.
    \item \textbf{Legal Reasoning Quality:} LARK scores 4.9/5 for sound legal logic and proper license interpretation, compared to 2.9/5 for LiDetector and 1.8/5 for rule-based tools. The evaluator highlighted LARK's ability to explain complex compatibility scenarios with clear legal precedents and regulatory context.
    \item \textbf{Actionable Recommendations:} LARK achieves 4.7/5 for practical solutions and compliance guidance, compared to 3.1/5 for LiDetector and 2.0/5 for rule-based tools. The evaluator appreciated specific remediation steps and alternative licensing strategies.
    \item \textbf{Overall Usefulness:} LARK receives 4.8/5 for general utility and comprehensiveness, compared to 3.2/5 for LiDetector and 2.1/5 for rule-based tools. The evaluator noted that LARK explanations provide sufficient detail for legal decision-making and regulatory compliance documentation.
\end{itemize}

\textbf{LLM Evaluator Performance:} The fine-tuned evaluator demonstrates high accuracy with 95.2\% agreement with expert ratings across all dimensions. Automated metrics confirm expert assessments: LARK provides 5.7 citations per explanation vs 1.8 for LiDetector, achieves 96\% citation precision vs 67\% for LiDetector, generates comprehensive explanations (average 387 words) vs 156 words for LiDetector, and maintains 98.1\% technical accuracy vs 93.2\% for LiDetector.

\textcolor{black}{\textbf{RQ$_3$} demonstrates that RAG-enhanced explanations provide superior quality across all evaluation dimensions while simultaneously mitigating hallucinations. The system generates explanations that include precise citations (5.7 per explanation vs 1.8 for baselines), comprehensive legal reasoning (4.8/5 expert rating vs 3.2/5), and actionable compliance recommendations, with significantly reduced hallucination rate due to KG constraints and citation grounding.}

\begin{table}[ht]
\centering
\caption{LLM-Based Explainability Evaluation Results}
\label{tab:explanation-quality}
\resizebox{0.9\linewidth}{!}{%
\begin{tabular}{lcccc}
\toprule
\textbf{Tool} & \textbf{Explainability} & \textbf{Citation Count} & \textbf{Citation Relevance} & \textbf{Legal Reasoning} \\
\midrule
FOSSology & Low & 0.2 ± 0.1 & 35\% ± 10\% & 1.5/5 ± 0.3 \\
ScanCode & Medium & 0.5 ± 0.2 & 45\% ± 12\% & 2.1/5 ± 0.4 \\
Ninka & Low & 0.1 ± 0.1 & 25\% ± 8\% & 1.2/5 ± 0.3 \\
FLICT & Medium & 1.8 ± 0.3 & 78\% ± 6\% & 3.5/5 ± 0.3 \\
LiDetector & Low & 1.8 ± 0.4 & 67\% ± 8\% & 2.9/5 ± 0.4 \\
OSS-LCAF & Medium & 2.1 ± 0.3 & 72\% ± 7\% & 3.2/5 ± 0.3 \\
ClauseBench & Medium & 1.9 ± 0.3 & 69\% ± 8\% & 3.1/5 ± 0.3 \\
SCANOSS & Low & 0.8 ± 0.2 & 52\% ± 10\% & 2.2/5 ± 0.4 \\
ContractEval & High & 3.2 ± 0.5 & 85\% ± 5\% & 3.8/5 ± 0.2 \\
LiCoEval & Medium & 2.8 ± 0.4 & 78\% ± 6\% & 3.1/5 ± 0.4 \\
LicenseGPT & High & 4.1 ± 0.6 & 89\% ± 4\% & 3.9/5 ± 0.2 \\
L3icNexus & High & 3.8 ± 0.5 & 87\% ± 5\% & 3.7/5 ± 0.3 \\
\textbf{LARK} & \textbf{High} & \textbf{5.7 ± 0.8} & \textbf{96\% ± 3\%} & \textbf{4.9/5 ± 0.1} \\
\bottomrule
\end{tabular}
}
\end{table}

\textbf{LLM Evaluator Validation:} Expert validation on 200 randomly selected explanations confirms 95.2\% agreement with LLM evaluator ratings across all dimensions (Cronbach's alpha > 0.85), demonstrating the reliability of our automated evaluation approach.

% \end{quote}





\begin{boxK}
\textit{\textbf{Summary for RQ$_3$.} RAG + KG yields high-quality, citation-backed explanations (5.7 citations/explanation; 96\% relevance; 4.9/5 legal reasoning) and markedly lowers hallucination compared to all 12 baselines.}
\end{boxK}


\end{comment}


\textbf{RQ$_4$: How effective is RAG in  mitigating hallucinations compared to all baselines?}


% RQ on hallucination is merged into RQ3 above
\noindent\textbf{Approach.} We evaluated LARK's hallucination mitigation capabilities by analyzing how the framework constrains LLM decision-making through structured knowledge graph outputs and RAG-grounded explanations. We compared LARK's constrained approach against unconstrained LLM responses and measured hallucination rates across different scenarios including novel license combinations, ambiguous compatibility cases, and edge cases not covered in training data.

\noindent\textbf{Results.} LARK demonstrates exceptional hallucination mitigation through its dual-constraint architecture. According to Table \ref{tab:hallucination-mitigation}, the framework achieves a 2.1\% hallucination rate compared to 18.7\% for unconstrained LLM responses and 8.3\% for traditional ML approaches. The knowledge graph component provides structured compatibility decisions that the LLM cannot override, ensuring factual accuracy in compatibility determinations. The RAG system grounds all explanations in authoritative legal sources, reducing hallucination in explanatory text by 89\% compared to LLM-only approaches.

The constrained architecture operates through two key mechanisms: (1) \textit{Knowledge Graph Constraint}: The LLM receives pre-computed compatibility results from the knowledge graph and is instructed to explain these results rather than generate new compatibility decisions, eliminating the possibility of hallucinated compatibility judgments; (2) \textit{RAG Grounding}: All explanations are generated using retrieved legal text chunks as context, ensuring that explanations reference actual legal sources rather than generating fictional legal reasoning. Expert evaluation confirms that 97.8\% of LARK's responses contain verifiable citations, compared to 23.4\% for unconstrained LLM responses.

%\textcolor{black}{\textbf{RQ$_6$} demonstrates that LARK's architecture effectively mitigates LLM hallucination through structured constraints and authoritative grounding. The knowledge graph provides factual compatibility decisions that cannot be hallucinated, while the RAG system ensures all explanations are grounded in real legal sources. This dual-constraint approach is essential for regulatory compliance where accuracy and verifiability are paramount, addressing a critical limitation of current LLM-based legal analysis tools.}

\begin{table}[ht]
\centering
\caption{Hallucination Mitigation Analysis}
\label{tab:hallucination-mitigation}
\resizebox{0.9\linewidth}{!}{%
\begin{tabular}{lcccc}
\toprule
\textbf{Approach} & \textbf{Hallucination Rate} & \textbf{Verifiable Citations} & \textbf{Factual Accuracy} & \textbf{Expert Confidence} \\
\midrule
Unconstrained LLM & 18.7\% & 23.4\% & 67.2\% & 2.1/5 \\
Traditional ML & 8.3\% & 45.6\% & 78.9\% & 3.2/5 \\
LARK (Full) & \textbf{2.1\%} & \textbf{97.8\%} & \textbf{96.2\%} & \textbf{4.8/5} \\
LARK - KG Constraint & 12.4\% & 89.3\% & 87.1\% & 3.9/5 \\
LARK - RAG Grounding & 6.7\% & 94.2\% & 91.8\% & 4.2/5 \\
\bottomrule
\end{tabular}
}
\end{table}

\begin{boxK}
\textit{\textbf{Summary for RQ$_4$.} LARK achieves exceptional hallucination mitigation with 2.1\% hallucination rate (vs 18.7\% for unconstrained LLM) through dual-constraint architecture: knowledge graph provides factual compatibility decisions that LLM cannot override, while RAG system grounds all explanations in authoritative legal sources with 97.8\% verifiable citations.}
\end{boxK}


%\vspace{-.2cm}
\section{Threats To Validity}
\label{Section:Threats}

\textbf{External Validity.} Our evaluation centers on open-source projects from GitHub repositories. Consequently, our findings may not generalize to all proprietary or commercially developed projects, particularly those with unique licensing arrangements or enterprise-specific license management practices. Although our dataset covers diverse domains (web development, machine learning, system utilities, developer tools), results may not extend to specialized domains with unique licensing requirements such as healthcare, automotive, or aerospace software. The custom license evaluation focused on enterprise licenses but may not represent all possible custom license variants encountered in practice. Our framework's LLM component relies on GPT-4, and performance may vary with other language models. Future research should evaluate LARK with alternative LLMs such as Claude, Gemini, or domain-specific models fine-tuned for legal text analysis.

\textbf{Internal and Construct Validity.} Concerning license compatibility ground truth, we established our evaluation dataset through expert validation and comparison with established compatibility matrices (OSADL, SPDX). However, license compatibility can be context-dependent and subject to legal interpretation variations. Our expert evaluation involved three legal professionals, but broader expert consensus might yield different explainability ratings. The custom license parsing evaluation focused on extracting obligations, prohibitions, and permissions, but may not capture all nuanced legal concepts present in complex proprietary licenses. Additionally, our Knowledge Graph construction relies on authoritative sources (OSI, SPDX), but license interpretations evolve over time, potentially affecting long-term accuracy. We mitigated this through the system's update capabilities, but continuous validation against emerging legal precedents remains necessary.




%\section{Discussion}
\label{Section:Discussion}
The LARK framework represents a significant advancement in automated license compatibility detection by integrating Knowledge Graphs, Large Language Models, and Retrieval-Augmented Generation. Our comprehensive evaluation demonstrates that this integrated approach addresses critical limitations in existing tools while providing superior performance across accuracy, explainability, and operational efficiency dimensions. This section discusses key insights from our research and implications for future license compliance systems.



\noindent\textbf{ Takeaway \#1: \textit{Knowledge Graph integration is essential for comprehensive license compatibility analysis.}} Our evaluation demonstrates that the Knowledge Graph component contributes critically to overall system effectiveness, providing 8.7\% accuracy improvement over LLM-only approaches. The structured representation across 750+ licenses and 20,000+ dependencies enables transitive compatibility analysis across complex dependency chains and supports rapid updates when new licenses emerge. The graph-based approach allows for systematic reasoning about license relationships that would be difficult to achieve through pattern matching or ML classification alone. Results from \textbf{RQ$_1$} and \textbf{RQ$_2$} show that removing the Knowledge Graph component reduces license detection accuracy to 89.4\% and fails compatibility analysis for 23\% of custom license combinations.

\noindent\textbf{ Takeaway \#2: \textit{LLM-powered custom license parsing enables unprecedented extensibility.}} Traditional license analysis tools struggle with custom and proprietary licenses due to their reliance on predefined patterns or trained models that require extensive retraining. Our LLM-based parsing approach achieves 94\% accuracy in processing custom licenses with 2.3-second processing time, compared to 0\% capability for rule-based tools. As demonstrated in \textbf{RQ$_2$}, the few-shot learning approach with 2-3 examples enables automatic adaptation to novel license formats, extracting obligations, prohibitions, and permissions without requiring model retraining. This capability is essential for enterprise environments where custom licenses are common but existing tools provide no support.



\noindent\textbf{ Takeaway \#3: \textit{RAG-enhanced explanations are crucial for regulatory compliance and legal decision-making.}} Expert evaluation reveals that LARK's RAG-enhanced explanations significantly outperform existing approaches, achieving 4.8/5 rating compared to 3.2/5 for LiDetector and 2.1/5 for rule-based tools. The system provides 3.2x more relevant citations per explanation (5.7 vs 1.8 for baselines) with 96\% citation relevance. As demonstrated in \textbf{RQ$_3$}, the RAG system retrieves relevant legal text chunks from 25,000+ indexed segments across comprehensive legal literature including authoritative books, IEEE/ACM articles, and regulatory guidelines with 92.8\% precision, enabling explanations that reference specific license clauses, legal precedents, regulatory requirements, and authoritative legal sources with exact page numbers and section references. This explainability capability addresses a critical gap in existing tools where binary compatibility decisions lack detailed justification, creating significant challenges for regulatory compliance where audit trails and justification documentation are essential.

\noindent\textbf{ Takeaway \#4: \textit{Dual-constraint architecture effectively mitigates LLM hallucination in legal analysis.}} As demonstrated in \textbf{RQ$_4$}, LARK's architecture addresses a critical limitation of LLM-based legal tools through structured constraints that prevent hallucination. The knowledge graph provides factual compatibility decisions that the LLM cannot override, ensuring that compatibility judgments are based on structured relationships rather than generated content. The RAG system grounds all explanations in authoritative legal sources, reducing hallucination in explanatory text by 89\% compared to LLM-only approaches. This dual-constraint approach achieves a 2.1\% hallucination rate compared to 18.7\% for unconstrained LLM responses, with 97.8\% of responses containing verifiable citations. This architecture is essential for regulatory compliance where accuracy and verifiability are paramount, addressing concerns about LLM reliability in legal decision-making contexts.

   % \noindent\textbf{ Takeaway \#7: \textit{There is potential data leakages between data used to train LLMs and data used to evaluate research. This must be controlled for/mitigated in research studies.}} %\eman{This is an implication (not takeaway), and it might sounds generic, would you recommend keeping it or removing it?}.}} 
   % \textcolor{red}{%The manner of LLM expressions of confidence or apologies reported in \textbf{RQ$_5$} may impact the emergence of certain types of issues, like data leakage, in software development projects. 
   %  A recent study has shed light on the potential risks associated with explicit or implicit data leakage between LLMs training data and research evaluation \cite{sallou2023breaking}. One identified threat is the possibility of data leakage, which arises from the blurred separation of training, validation, and test sets \cite{yang2022data}. While this concern is recognized in LLM-related research, it is particularly relevant in the context of utilizing LLMs for software engineering tasks, such as license analysis. To address this challenge, we propose a recommendation for future studies using LLMs for license compatibility tasks. Specifically, researchers should consider using recent GitHub metadata, including license-related commits, issues, pull requests, and other relevant data, as a test dataset. This recommendation is based on recent findings that explored the types of tasks generated by LLMs \cite{tufano2024unveiling}. By doing so, researchers can mitigate the risk of data leakage, as the model behind LLMs is unlikely to have encountered these data during its training phase, ensuring data integrity and reliability.}

  
\section{Conclusions}
\label{Section:Conclusions}

\textcolor{black}{This paper presents LARK (License Analysis with RAG and Knowledge graphs), a novel framework that addresses critical challenges in software license compatibility detection and regulatory compliance. Modern software development increasingly relies on complex ecosystems of open-source components, where 72.91\% of projects encounter license incompatibilities that threaten project viability and regulatory compliance. Our methodology integrates Knowledge Graphs, Large Language Models, and Retrieval-Augmented Generation to provide comprehensive license compatibility analysis with explainable reasoning.

\textbf{Technical Contributions:} Our framework makes several key technical contributions: (1) \textit{Integrated Knowledge Representation} through Neo4j-based knowledge graphs that model 750+ licenses, 20,000+ dependencies, and their compatibility relationships enabling transitive analysis across complex dependency chains; (2) \textit{LLM-Powered Custom License Integration} using GPT-4 with few-shot learning to automatically parse and integrate novel license texts without requiring model retraining; (3) \textit{RAG-Enhanced Explainability} that provides detailed, citation-backed explanations with 92.8\% retrieval precision from 25,000+ indexed segments across comprehensive legal literature including authoritative books, IEEE/ACM articles, and regulatory guidelines, providing 3.2x more citations than existing approaches with exact page numbers and section references; (4) \textit{Dual-Constraint Hallucination Mitigation} through knowledge graph constraints that prevent LLM decision-making and RAG grounding that ensures all explanations reference authoritative legal sources, achieving 2.1\% hallucination rate compared to 18.7\% for unconstrained LLM responses; (5) \textit{Dynamic Update Capabilities} supporting rapid adaptation to evolving legal interpretations through graph-based knowledge representation.

\textbf{Experimental Validation:} Our comprehensive evaluation on 4,000 OSS projects demonstrates significant improvements over existing approaches. LARK achieves (1) 98.1\% license detection accuracy compared to 93.2\% for LiDetector; (2) 96.2\% conflict detection F1 score versus 88.7\% for baselines; (3) 4.8/5 explainability score compared to 3.2/5 for existing tools; (4) 94\% accuracy in processing custom and proprietary licenses where traditional tools provide 23\% capability; (5) 2.1\% hallucination rate compared to 18.7\% for unconstrained LLM responses, with 97.8\% verifiable citations; (6) 24-hour update cycles versus 168 hours for ML-based approaches; (7) 0.32GB memory usage representing 71\% reduction compared to existing systems.

\textbf{Practical Impact:} The framework addresses critical gaps in existing approaches: limited explainability in current tools, static knowledge representation requiring manual updates, narrow compatibility analysis focus, and poor custom license handling. LARK provides citation-backed explanations essential for regulatory compliance, supports automatic integration of custom licenses, and enables real-time compatibility monitoring through CI/CD integration.

\textbf{Future Work:} We plan to extend coverage to emerging license formats, integrate additional package managers, develop regulatory framework-specific policies (GDPR, CCPA), and create industry-specific knowledge graphs for domain-specific compliance requirements. Further research directions include fine-tuning specialized legal LLMs, developing temporal license evolution models, and exploring multi-modal analysis combining license text with code patterns.

We believe that our results contribute to a significant advancement in automated license compatibility detection, providing developers and organizations with accurate, explainable, and scalable solutions for legal risk mitigation and regulatory compliance in modern software engineering.

%\noindent \textbf{Acknowledgments.} We would like to thank the reviewers at IST for their detailed and invaluable feedback.

\noindent \textbf{Data Availability Statement.} The data is publicly available at \cite{ReplicationPackage}.

\noindent{\textbf{Declaration of generative AI and AI-assisted technologies in the writing process.}}



\section{Acknowledgements}
\label{sec:Acknowledgements}
We would like to thank our industrial partner Vermeg-Tunisia for providing the resources and support necessary to conduct this research. Their assistance in 
coordinating the data acquisition and labeling process was invaluable.
This study has been funded by the Tunisian Young Researchers’ Encouragement Program (Ed. 2022) (22PEJC-D3P2).\\

\section{Conflict of interest}
All authors in this paper declared that there is no conflicts of interest to this work.



\bibliography{references}


\appendix
\section{End-to-End Example}
\label{APP1}
Here we include an example that supports the main text.
Consider the user query: \textit{"I'm building a commercial web application using React, Express.js, and MongoDB. Are there any license conflicts I should be aware of?"}

\textbf{Step 1 - Query Processing:}
\begin{verbatim}
Extracted Dependencies: ["react", "express", "mongodb"]
Query Type: Compatibility Check
Context: Commercial Use
\end{verbatim}

\textbf{Step 2 - Knowledge Graph Query:}
\begin{verbatim}
MATCH (p1:Package {name: "react"})-[:HAS_LICENSE]->(l1:License)
MATCH (p2:Package {name: "express"})-[:HAS_LICENSE]->(l2:License)  
MATCH (p3:Package {name: "mongodb"})-[:HAS_LICENSE]->(l3:License)
MATCH (l1)-[r1:IS_COMPATIBLE_WITH]->(l2)
MATCH (l2)-[r2:IS_COMPATIBLE_WITH]->(l3)
RETURN p1.name, l1.name, l2.name, l3.name, 
       r1.compatibility_type, r2.compatibility_type
\end{verbatim}

\textbf{Step 3 - RAG Retrieval:}
The system retrieves relevant legal documents about MIT License (React), MIT License (Express), and Server Side Public License (MongoDB) commercial use provisions.

\textbf{Step 4 - Response Generation:}
\begin{quote}
\textbf{License Compatibility Analysis:}

\textbf{Dependencies Found:}
\begin{itemize}
    \item React: MIT License
    \item Express.js: MIT License
    \item MongoDB: Server Side Public License (SSPL)
\end{itemize}

\textbf{Compatibility Status:} \checkmark \textbf{COMPATIBLE}

\textbf{Analysis:} All three dependencies are compatible for commercial use. React and Express both use permissive MIT licenses that allow commercial use without restrictions. MongoDB's SSPL requires that if you offer MongoDB as a service, you must open-source your service implementation, but this doesn't affect typical web applications.

\textbf{Legal Basis:} This analysis is supported by the MIT License terms (Section 1) and SSPL Section 13, as referenced in \textit{Open Source Initiative License Review} (2021) and \textit{MongoDB Inc. v. Amazon Web Services} (2020).

\textbf{Recommendations:}
\begin{itemize}
    \item No license conflicts detected
    \item Commercial use is permitted
    \item Consider MongoDB's SSPL implications if offering database-as-a-service
\end{itemize}
\end{quote}



\section{Cypher query generation}
\label{APP2}


For the query "Can I use React with Apache 2.0?", the system generates:

\begin{verbatim}
MATCH (p1:Package {name: "react"})-[:HAS_LICENSE]->(l1:License)
MATCH (l2:License {name: "Apache-2.0"})
MATCH (l1)-[r:IS_COMPATIBLE_WITH]->(l2)
RETURN p1.name, l1.name, l2.name, r.compatibility_type, 
       r.confidence_score, r.notes
\end{verbatim}



\section{Cypher query for compatibility relationship creation}
\label{APP3}

\begin{verbatim}
MATCH (l1:License {spdx_id: $license1}), (l2:License {spdx_id: $license2})
CREATE (l1)-[:COMPATIBLE_WITH {confidence: $confidence, 
                                method: 'LLM_parsing', 
                                terms_analyzed: $terms}]->(l2)
\end{verbatim}



\end{document}



